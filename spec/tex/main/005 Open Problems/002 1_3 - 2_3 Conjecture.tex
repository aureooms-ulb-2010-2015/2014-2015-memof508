\section{$\sfrac{1}{3}$--$\sfrac{2}{3}$ Conjecture}

Quoting from \cite{wiki:linext},

\begin{quotation}
The $\sfrac{1}{3}$--$\sfrac{2}{3}$ conjecture states that in any finite partially ordered set $P$ that is not totally ordered there exists a pair $(x,y)$ of elements of $P$ for which the linear extensions of $P$ in which $x < y$ number between $\sfrac{1}{3}$ and $\sfrac{2}{3}$ of the total number of linear extensions of $P$. An equivalent way of stating the conjecture is that, if one chooses a linear extension of $P$ uniformly at random, there is a pair $(x,y)$ which has probability between $\sfrac{1}{3}$ and $\sfrac{2}{3}$ of being ordered as $x < y$.
\end{quotation}

Note that the conjecture has already been proved for other constants than $\sfrac{1}{3}$--$\sfrac{2}{3}$ (\cite{kahn1984balancing},\cite{linial1984information}, \cite{kahn1991balancing}, \cite{brightwell1995balancing} and \cite{brightwell1999balanced}). But the question still remains and as far as we know the $\sfrac{1}{3}$ bound is tight since we have examples of width-$2$ extreme case posets for which the bound is exact. 


Their are some results for specific classes of partial orders verifying the conjecture. Since the proportion of partial orders contained in those classes approaches $1$ as $n$ grows, we can conclude that the proportion of $n$-element partial orders obeing the $\sfrac{1}{3}$--$\sfrac{2}{3}$ conjecture approaches $1$.


References : \cite{kral2013new}, \cite{zaguia20111}, \cite{peczarski2006gold}, \cite{peczarski2008gold}.