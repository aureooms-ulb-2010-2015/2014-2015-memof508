\section{Chains and antichains}

A \concept{chain} (or \concept{totally ordered set} or \concept{linearly ordered set}) is a poset in which any two elements are comparable. Thus the poset $n$ of \ref{ex:poset:def:a} is a chain. A subset $C$ of a poset $P$ is called a chain if $C$ is a chain when regarded as a subposet of $P$. The chain $C$ of $P$ is called maximal if it is not contained in a larger chain of $P$. The chain $C$ of $P$ is called saturated (or unrefinable) if there does not exist $u \in P - C$ such that $s < u < t$ for some $s, t \in C$ and such that $C \cup \left\{{u}\right\}$ is a chain. Thus maximal chains are saturated, but not conversely. In a locally finite poset, a chain $t_0 < t_1 < \cdots < t_n$ is saturated if and only if $t_{i-1} \lessdot t_i$ for $1 \leq i \leq n$. The length $\ell(C)$ of a finite chain is defined by $\ell(C) = \#C - 1$. The length (or rank) of a finite poset $P$ is

$$\ell(P) := max\left\{{\ell(C) : C ~\text{is a chain of}~ P}\right\}.$$

The length of an interval $[s, t]$ is denoted $\ell(s, t)$. If every maximal chain of $P$ has the same length $n$, then we say that $P$ is graded of rank $n$. In this case there is a unique rank function $\rho : P \to \left\{{0, 1, \cdots , n}\right\}$ such that $\rho(s) = 0$ if $s$ is a minimal element of $P$, and $\rho(t) = \rho(s) + 1$ if $t \gtrdot s$ in $P$. If $s \leq t$ then we also write $\rho(s, t) = \rho(t) - \rho(s) = \ell(s, t)$. If $\rho(s) = i$, then we say that $s$ has rank $i$.




An \concept{antichain} (or \concept{Sperner family} or \concept{clutter}) is a subset $A$ of a poset $P$ such that any two distinct elements of $A$ are incomparable. An \concept{order} ideal (or \concept{semi-ideal} or \concept{down-set} or \concept{decreasing subset}) of $P$ is a subset $I$ of $P$ such that if $t \in I$ and $s \leq t$, then $s \in I$. Similarly a \concept{dual order ideal} (or \concept{up-set} or \concept{increasing subset} or \concept{filter} ) is a subset $I$ of $P$ such that if $t \in I$ and $s \geq t$, then $s \in I$. When $P$ is finite, there is a one-to-one correspondence between antichains $A$ of $P$ and order ideals $I$. Namely, $A$ is the set of maximal elements of $I$, while

$$I = \left\{{s \in P : s \leq t ~\text{for some t}~ \in A}\right\}.$$

The set of all order ideals of $P$, ordered by inclusion, forms a poset denoted $J(P)$. In Section 3.4 we shall investigate $J(P)$ in greater detail. If $I$ and $A$ are related as in equation (3.2), then we say that $A$ \concept{generates} $I$. If $A = \left\{{t_1 , \cdots , t_k}\right\}$, then we write $I = \langle t_1 , \cdots , t_k \rangle$ for the order ideal generated by $A$. The order ideal $\langle t \rangle$ is the \concept{principal order ideal} generated by $t$, denoted $\Lambda_t$. Similarly $V_t$ denotes the principal dual order ideal generated by $t$, that is, $V_t = \left\{{s \in P : s \geq t}\right\}$.

