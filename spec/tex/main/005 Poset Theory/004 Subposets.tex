\section{Subposets}

Some care has to be taken in defining the notion of “subposet.” By a weak subposet of $P$, we mean a subset $Q$ of the elements of $P$ and a partial ordering of $Q$ such that if $s \leq t$ in $Q$, then $s \leq t$ in $P$ . If $Q$ is a weak subposet of $P$ with $P = Q$ as sets, then we call $P$ a refinement of $Q$. By an induced subposet of $P$, we mean a subset $Q$ of $P$ and a partial ordering of $Q$ such that for $s, t \in Q$ we have $s \leq t$ in $Q$ if and only if $s \leq t$ in $P$ . We then say the subset $Q$ of $P$ has the induced order. Thus the finite poset $P$ has exactly $2^{\#P}$ induced subposets. By a subposet of $P$, we will always mean an induced subposet. A special type of subposet of $P$ is the (closed) interval $[s, t] = \left\{{u \in P : s \leq u \leq t}\right\}$, defined whenever $s \leq t$. (Thus the empty set is not regarded as a closed interval.) The interval $[s, s]$ consists of the single point $s$. We similarly define the open interval $(s, t) = \left\{{u \in P : s < u < t}\right\}$, so $(s, s) = \emptyset$. If every interval of P is finite, then P is called a locally finite poset. We also define a subposet $Q$ of $P$ to be convex if $t \in Q$ whenever $s < t < u$ in $P$ and $s, u \in Q$. Thus an interval is convex. \cite{Stanley:2011:ECV:2124415}