\section{Definition}

\concept{poset} stands for \emph{Partially Ordered Set}. A partially ordered set $P$ is a set (which by abuse of notation we also call $P$), together with a binary relation denoted $\leq$ (or $\leq_P$ when there is a possibility of confusion), satisfying the following three axioms \cite{Stanley:2011:ECV:2124415}:

\begin{enumerate}
\item For all $t \in P, t \leq t$ (reflexivity).
\item If $s \leq t$ and $t \leq s$, then $s = t$ (antisymmetry).
\item If $s \leq t$ and $t \leq u$, then $s \leq u$ (transitivity).
\end{enumerate}

We say that two elements $s$ and $t$ of $P$ are comparable if $s \leq t$ or $t \leq s$; otherwise, $s$ and $t$ are incomparable, denoted $s \parallel t$.

We use the obvious notation $t \geq s$ to mean $s \leq t$, $s < t$ to mean $s \leq t$ and $s \neq t$, and $t > s$ to mean $s < t$.
