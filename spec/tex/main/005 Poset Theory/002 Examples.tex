\section{Examples}

\nb{For everything that follows $[n] = \left\{{1, 2, \cdots, n}\right\}$}

\begin{example}
\label{ex:poset:def}

\item \label{ex:poset:def:a} Let $n \in \mathbb{P}$. The set $[n]$ with its usual order forms an $n$-element poset with the special property that any two elements are comparable. This poset is denoted $n$. Of course $n$ and $[n]$ coincide as sets, but we use the notation $n$ to emphasize the order structure. \cite{Stanley:2011:ECV:2124415}

\item \label{ex:poset:def:b} Let $n \in \mathbb{N}$. We can make the set $2^{[n]}$ of all subsets of $[n]$ into a poset $B_n$ by defining $S \leq T$ in $B_n$ if $S \subseteq T$ as sets. One says that $B_n$ consists of the subsets of $[n]$ ``ordered by inclusion''. \cite{Stanley:2011:ECV:2124415}

\item \label{ex:poset:def:c}  Let $n \in \mathbb{P}$. The set of all positive integer divisors of $n$ can be made into a poset $D_n$ in a natural way by defining $i \leq j$ in $D_n$ if $j$ is divisible by $i$ (denoted $i \mid j$). \cite{Stanley:2011:ECV:2124415}

\item \label{ex:poset:def:d}  Let $n \in \mathbb{P}$. We can make the set $\Pi_n$ of all partitions of $[n]$ into a poset (also denoted $\Pi_n$) by defining $\pi \leq \sigma$ in $\Pi_n$ if every block of $\pi$ is contained in a block of $\sigma$. For instance, if $n = 9$ and if $\pi$ has blocks $137, 2, 46, 5, 9$, and $\sigma$ has blocks $13467, 2589$, then $\pi \leq \sigma$. We then say that $\pi$ is a \emph{refinement} of $\sigma$ and that $\Pi_n$ consists of the partitions of $[n]$ ``ordered by refinement''. \cite{Stanley:2011:ECV:2124415}
\end{example}