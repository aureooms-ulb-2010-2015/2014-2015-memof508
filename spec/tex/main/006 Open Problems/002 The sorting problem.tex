\section{The sorting problem}


\subsection{Average case analysis of shell sort}


Find whether or not there exist a gap sequence that makes the algorithm run in $\BigO{n \log n}$.


References : \cite{ciura2001best}, \cite{jiang2000lower}, \cite{sedgewick1996analysis}, \cite{knuth1998art}.



\subsection{Problem 41: Sorting $X + Y$ (Pairwise Sums)}

\url{http://cs.smith.edu/~orourke/TOPP/P41.html#Problem.41}


Given two sets of numbers, each of size $n$, how quickly can the set of all pairwise sums be sorted? In symbols, given two sets $X$ and $Y$, our goal is to sort the set

$$ X + Y = \left\{{x + y \mid x \in X, y \in Y }\right\} $$

The earliest known reference is Fredman \cite{fredman1976good}, who attributes the problem to Elwyn Berlekamp.

This is a simple special case of the more general question of sorting with partial information: How many comparisons are required to sort if a partial order on the input set is already known? Hernández Barrera \cite{barrera1996finding} and Barequet and Har-Peled \cite{barequet2001polygon} describe several geometric problems that are ``Sorting-($X + Y$)-hard''. Specifically, there is a subquadratic-time transformation from sorting $X + Y$ to each of the following problems: computing the Minkowski sum of two orthogonal-convex polygons, determining whether one monotone polygon can be translated to fit inside another, determining whether one convex polygon can be rotated to fit inside another, sorting the vertices of a line arrangement, or sorting the interpoint distances between $n$ points in $\mathbb{R}^d$. (Although Barequet and Har-Peled \cite{barequet2001polygon} claim only that the problems they consider are \emph{3SUM-hard}, their proofs immediately imply this stronger result.) Fredman also mentions an immediate application to multiplying sparse polynomials \cite{fredman1976good}.

More references : \cite{kahnkim1}, \cite{dietzfelbinger1989lower}, \cite{steiger1995pseudo}, \cite{lambert1990sorting}, \cite{erickson1997lower}, \cite{bremner2012necklaces}.







\subsection{Pancake sorting}


Pancake sorting is the colloquial term for the mathematical problem of sorting a disordered stack of pancakes in order of size when a spatula can be inserted at any point in the stack and used to flip all pancakes above it. ``Pancake number'' refers to the maximum number of flips required for a given number of pancakes. In this form, the problem was first discussed by American geometer Jacob E. Goodman.[1] It is a variation of the sorting problem in which the only allowed operation is to reverse the elements of some prefix of the sequence. Unlike a traditional sorting algorithm, which attempts to sort with the fewest comparisons possible, the goal is to sort the sequence in as few reversals as possible. A variant of the problem is concerned with burnt pancakes, where each pancake has a burnt side and all pancakes must, in addition, end up with the burnt side on bottom.


References : \cite{cibulka2011average}, \cite{cohen1995problem}.