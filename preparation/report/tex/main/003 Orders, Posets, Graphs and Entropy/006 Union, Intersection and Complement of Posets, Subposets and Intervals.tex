\section{Union, Intersection and Complement of Posets, Subposets and Intervals}
\label{tree:poset:sub}


Since a poset is the combination $(S, \le_S)$ of a set $S$ and an order $\le_S$, we might be interested to apply standard set operations to posets.

The binary operations Union ($\cup$), Intersection ($\cap$) and Complement ($\setminus$) for posets $P = (S, \le_S)$ and $Q = (T, \le_T)$ only make sense when $\le_S = \le_T$. For example, when we run the algorithm merge sort at each recursion step we merge (read ``make the union of'') two posets having the same total order. We thus only define those operations for posets having the same order relation, and the result of any of those three binary operations on $P$ and $Q$ is a new poset $(U, \le_U)$ where $U = S~\text{op}~T$ and $\le_U = \le_S = \le_T$.


In the same way we handled the binary operations we might look at how we can define what a subposet is. For what follows, $P = (S, \le_S)$ is a poset and $Q = (T, \le_T)$ a subposet of $P$.

Unlike the binary operations, we might consider the case where $\le_T \neq \le_S$. Two main types of subposets emerge: subposets for which $\le_T = \le_S$ and others. The former will be called induced subposets and the later weak subposets.

Quoting from \cite{Stanley:2011:ECV:2124415},

\begin{quotation}

By an induced subposet of $P$, we mean a subset $Q$ of $P$ and a partial ordering of $Q$ such that for $s, t \in Q$ we have $s \leq t$ in $Q$ if and only if $s \leq t$ in $P$. We then say the subset $Q$ of $P$ has the induced order.

\end{quotation}

This definition is similar to the definition of an induced subgraph. If you represent a poset $P$ as a directed graph $G = (V, E)$ (see \ref{tree:poset:graph}), each edge represents an element of an order relation and when you remove a vertex from one of those graphs you also have to remove incident edges since one of their endpoints is missing. Considering that $P$ is finite, there are $2^{\#V}$ induced subgraphs of $G$, thus the poset $P$ has exactly $2^{\#P}$ induced subposets.

\nb{Note that when using the expression ``a subposet of $P$'', we will always mean ``an induced subposet of $P$''.}

The set of weak subposets is a superset of the set of induced subposets where arbitrary edge deletion in the graph associated with a poset is allowed.

\nb{All induced subposets are also weak subposets.}

A weak subposet of $P$ is thus a subset $Q$ of the elements of $P$ and a partial ordering of $Q$ such that if $s \leq t$ in $Q$, then $s \leq t$ in $P$.

\nb{If $Q$ is a weak subposet of $P$ with $P = Q$ as sets, then we call $P$ a refinement of $Q$ (only the order relations differ).}


A special type of subposet of $P$ is the (closed) interval $[s, t] = \left\{{u \in P : s \leq u \leq t}\right\}$, defined whenever $s \leq t$. The interval $[s, s]$ consists of the single point $s$.

\nb{By this definition the empty set is not regarded as a closed interval since partial order relations are reflexive.}

Similarly we can define the open interval $(s, t) = \left\{{u \in P : s < u < t}\right\}$, so $(s, s) = \emptyset$.

If every interval of $P$ is finite, then $P$ is called a locally finite poset, e.g. $P = (\mathbb{N}, \le)$ is locally finite while $Q = (\mathbb{R}, \le)$ is not.


We also define a subposet $Q$ of $P$ to be convex if $t \in Q$ whenever $s < t < u$ in $P$ and $s, u \in Q$. Thus an interval is convex. The nuance with a closed interval is that the number of bounds can be greater than 2.

%TODO figure of convex, interval