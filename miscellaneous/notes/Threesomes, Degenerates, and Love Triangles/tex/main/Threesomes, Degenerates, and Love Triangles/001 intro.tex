

\section{The 3SUM Problem}

The problem is the following,

\begin{problem}
The 3\textsc{sum} problem consists in finding three numbers $a$, $b$ and $c$, for
$a \in A, b \in B, c \in C$ and where $A, B, C \subset \mathbb{R}$, that sum
up to $0$, i.e. $a + b + c = 0$.
\end{problem}


\section{The 3SUM Conjecture}

In a recent paper (\cite{DBLP:journals/corr/JorgensenP14}) it is proved that
the 3\textsc{sum} conjecture does not hold true. As a reminder,

\begin{conjecture}
The 3\textsc{sum} conjecture states that the lower bound on the number of
comparisons in the decision tree model for the 3\textsc{sum} problem
is \BigOmega{n^2}.
\end{conjecture}

It was indeed proved that this lower bound holds for a particular case of the
decision tree model. By restricting himself to a $3$-linear decision tree model,
Jeff Erickson proves that 3\textsc{sum} is \BigOmega{n^2} in this particular
model.


\section{Variations of the Decision Tree Model}

Note that for $3\textsc{sum}$ we consider a $3$-linear decision tree model
since the function we evaluate at each step is linear and has 3 variables.
Here the function is $f( a, b, c ) = a + b + c$ and we make decisions depending on
the sign of the output of this function ( $<0, =0, >0$ ).

More generally, in \cite{cj99-08}, Jeff Erickson proves that in the $k$-linear
decision tree model, for any $k > 0$, the $k\textsc{sum}$ problem,
whose definition can be easily deduced from the $3\textsc{sum}$ problem
definition, has a \BigOmega{n^{\sfrac{k}{2}}} lower bound for even $k$ and
\BigOmega{n^{\sfrac{(k+1)}{2}}} for odd $k$.


\section{Why The Fuss?}

Although the 3\textsc{sum} problem has not a lot of practical applications, many
other problems are reducible from 3\textsc{sum} and thus a lower bound for
3\textsc{sum} would imply lower bounds for those other problems.

One of these problem is the $3$-points-on-line problem,

\begin{problem}
In the plane, the $3$-points-on-line problem consists in finding
three colinear points $a$, $b$ and $c$, for $a, b and c \in S, |S| = n$.
\end{problem}

We can reduce 3\textsc{sum} to $3$-points-on-line by mapping all
$a \in A, b \in B, c \in C$ to the points $a' = (a, a^3)$, $b' = (b, b^3)$
and $c' = (c, c^3)$.

This works because if $a'$, $b'$ and $c'$ are colinear we have,

\begin{eqnarray*}
	a'_x ( b'_y - c'_y ) + b'_x ( c'_y - a'_y ) + c'_x ( a'_y - b'_y ) & = & 0 \\
	a ( b^3 - c^3 ) + b ( c^3 - a^3 ) + c ( a^3 - b^3 ) & = & 0 \\
\end{eqnarray*}

When $ a + b + c = 0 \iff c = - ( a + b )$ we have,

\begin{eqnarray*}
	a b^3 + a ( a^3 + 3 a^2 b + 3 a b^2 + b^3 ) & & \\
	- b ( a^3 + 3 a^2 b + 3 a b^2 + b^3 ) - a^3 b - a^4 - a^3 b + b^4 + a b^3 & = & 0 \\
	\mathunderline{cyan}{a b^3} + \mathunderline{green}{a^4} + \mathunderline{blue}{3 a^3 b} + \mathunderline{red}{3 a^2 b^2} + \mathunderline{cyan}{a b^3} - \mathoverline{blue}{a^3 b} - \mathoverline{red}{3 a^2 b^2} & & \\
	- \mathoverline{cyan}{3 a b^3} - \mathoverline{yellow}{b^4} - \mathoverline{blue}{a^3 b} - \mathoverline{green}{a^4} - \mathoverline{blue}{a^3 b} + \mathunderline{yellow}{b^4} + \mathunderline{cyan}{a b^3} & = & 0 \\
	0 & = & 0 \\
\end{eqnarray*}

\cite{DBLP:journals/comgeo/GajentaanO12} and \cite{king2004survey} propose a list
of 3\textsc{sum}-HARD problem, i.e. problems that can be reduced from 3\textsc{sum}.

It means that all those 3\textsc{sum}-HARD problems are doomed to have
a \BigOmega{n^2} lower bound.


\section{Subquadratic Algorithms and Thinking Outside of the Box}

In April 2014, Pettie and Grønlund released a paper that showed that there
exists an algorithm for 3\textsc{sum} making a subquadratic amount of evaluation
of the function $f( a, b, c, d ) \in \{ <0, =0, >0 \}$.

Here, the secret lies thus in using a $k+1$-linear decision tree instead of the
classical $k$-linear decision tree approach.

The first algorithm they propose in \cite{DBLP:journals/corr/JorgensenP14} solves
the problem using a clever multi-$X + Y$ sorting procedure.

In their algorithm, they partition a set $A$ in $\ceil{\sfrac{n}{g}}$ groups
$A_i$ for $i \in [0, \ceil{\sfrac{n}{g}}[$ of size at most $g$ and need to sort
each
$A_{i,j} \buildrel \text{d{}ef}\over = A_i + A_j \forall i, j \in [0, \ceil{\sfrac{n}{g}}[$
.

Using a simple trick attributed to Fredman i.e.
$a_i + a_j < b_i + b_j \iff a_i - b_i < b_j - a_j$ they manage to do only
\BigO{n \log n + gn} comparisons.


http://cs.smith.edu/~orourke/TOPP/P11.html

http://cs.smith.edu/~orourke/TOPP/P41.html
