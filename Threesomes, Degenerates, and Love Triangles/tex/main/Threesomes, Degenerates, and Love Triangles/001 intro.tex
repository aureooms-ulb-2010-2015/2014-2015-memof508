\section{Introduction to the subject}

The problem is the following,

\begin{problem}
The 3\textsc{sum} problem consists in finding three numbers $a$, $b$ and $c$, for
$a \in A, b \in B, c \in C$ and where $A, B, C \subset \mathbb{R}$, that sum
up to $0$, i.e. $a + b + c = 0$.
\end{problem}

In a recent paper (\cite{DBLP:journals/corr/JorgensenP14}) it is proved that
the 3\textsc{sum} conjecture does not hold true. As a reminder,

\begin{conjecture}
The 3\textsc{sum} conjecture states that the lower bound on the number of
comparisons in the decision tree model for the 3\textsc{sum} problem
is \BigOmega{n^2}.
\end{conjecture}

It was indeed proved that this lower bound holds for a particular case of the
decision tree model. By restricting himself to a $3$-linear decision tree model,
Jeff Erickson proves that 3\textsc{sum} is \BigOmega{n^2} in this particular
model.

In fact, in \cite{cj99-08}, he proves that in the $k$-linear decision tree model
, for any $k > 0$, the $k\textsc{sum}$ problem is \BigOmega{n^{\sfrac{k}{2}}}
for even $k$ and \BigOmega{n^{\sfrac{(k+1)}{2}}} for odd $k$.

Although the 3\textsc{sum} problem has not a lot of practical applications, many
other problems are reducible from 3\textsc{sum} and thus a lower bound for
3\textsc{sum} would imply lower bounds for those other problems.

In April 2014, Pettie and Grønlund \dots
