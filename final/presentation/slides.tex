\documentclass[10pt,xcolor=x11names,dvipsnames,hyperref={colorlinks=false,breaklinks=true,bookmarks=true}]{beamer}
\usepackage{stylesheet}
%\usepackage{pgfpages}
%\setbeameroption{show notes on second screen}

\begin{document}

% TITLE, AUTHOR, DATE, INSTITUTE
\title[]{\textbf{Lower Bounds and Algorithms\\ in the\\ Linear Decision Tree Model\vspace{3mm}}}
\author[]{\large OOMS AURÉLIEN\\ADVISOR: PROF. JEAN CARDINAL\vspace{5mm}}
\institute[]{%
\footnotesize MÉMOIRE PRÉSENTÉ EN VUE DE L'OBTENTION\\
DU DIPLÔME DU MASTER EN SCIENCES INFORMATIQUES\\
ANNÉE ACADÉMIQUE 2014~-~2015\\[3mm]
\small UNIVERSITÉ LIBRE DE BRUXELLES}
\date{}

\begin{frame}
\titlepage{}
\end{frame}

\section{Introduction}
\begin{frame}\frametitle{\insertsection}\justifying
We focus on studying the complexity of problems and solving problems
in the linear decision tree model. In the model we use, solving a problem
amounts to retrieve information. We are allowed to retrieve information
through an oracle by means of questions that
can be answered ``yes'' or ``no''.\\[7mm]\pause
\bblue{\large Outline}
\begin{itemize}[label={\color{prussianblue}$\bullet$},itemsep=6pt]
\item Decision Tree Model\pause
\item Information-Theoretic Lower Bound\pause
\item Sorting under Partial Information\pause
\item \(k\)-linear Degeneracy Testing\pause
\item Application of Meiser's Algorithm
\end{itemize}
\end{frame}

\section{Decision Tree Model}
\begin{frame}\frametitle{\insertsection}\justifying
\begin{defn}[(Decision Tree Model)]
We are allowed to ask questions to an oracle \(\le\) that are answered
``yes'' or ``no''. Hence, each answer gives us at most
one additional bit of information.
Each question asked to the oracle costs us a single unit.
Every other operation can be carried out for free.
\end{defn}\pause
The goal of each of our analyses is to show that at least a certain
number of questions are required to be asked to solve a given problem, or
to provide an algorithm solving this problem using at most a certain number of
questions. Those two kinds of results are called lower bounds and upper bounds
respectively. A lower bound or an upper bound is in general expressed as
a function of the input size.
\end{frame}

\section{Information-Theoretic Lower Bound}
\begin{frame}\frametitle{\insertsection}\justifying
\begin{defn}[(\textsc{ITLB})]
The \emph{information-theoretic lower bound} (\textsc{ITLB}) of a problem
is the minimal depth of any
decision tree solving this problem. Generally speaking, the value of this lower
bound is the logarithm\footnote{Throughout this presentation, \(\log x\) denotes the binary logarithm of \(x\).}
in base \(2\) of the number of feasible solutions for this problem.
\end{defn}\pause
This lower bound is our goal to attain in terms of number of questions
when designing an efficient algorithm solving a problem. If \(\Gamma\) is the
set of feasible solutions of our problem, the \textsc{ITLB} only means
that we cannot do better than \(\ceil{\log(\Gamma)}\) questions, but it does not
necessarily mean that there exists an algorithm or a decision tree achieving
this bound in terms of complexity.
\end{frame}


\section{Sorting under Partial Information}
\begin{frame}\frametitle{\insertsection}\justifying
\begin{probl}[Sorting under Partial Information (\textsc{SUPI})]
Let \(\S = \enum{s_1 , \ldots , s_n}\) be a set
equipped with an unknown linear order. Given a subset of the relations \(s_i
\leq s_j\), determine the complete linear order by queries of the form
\(s_i \ask{\leq} s_j\).
\end{probl}\pause

\begin{ex}[(\textsc{SUPI})]
\parbox[b][3cm][t]{.45\textwidth}{
\psset{unit=.95cm}
\begin{pspicture}(-0.5,2.32)(3.5,0.82)
{\psset{linewidth=1.0pt}
\begin{footnotesize}
\only<2->{\psset{labelsep=0pt}
\uput{3pt}[225](0.5,0.5){\(c\)}
\uput{0.2pt}[315](1.,0.5){\(d\)}
\uput{2.5pt}[135](0.5,1.){\(a\)}
\uput{2.5pt}[45](1.,1.){\(b\)}
}
\end{footnotesize}
\only<2->{\psset{linecolor=black}
\psline(0.5,0.5)(0.5,1.)
\psline(0.5,0.5)(1.,1.)
\psline(1.,1.)(1.,0.5)
\psdots(0.5,0.5)
\psdots(1.,0.5)
\psdots(0.5,1.)
\psdots(1.,1.)
}
\only<3->{{\psset{linestyle=dotted}
\psline(0.5,1.)(1.,1.)
}}
\only<4->{
\psline{->}(1.5,.75)(1.75,.75)
}
\begin{footnotesize}
\only<4->{\psset{labelsep=0pt}
\uput{3pt}[225](2,0.25){\(c\)}
\uput{0.2pt}[315](2.5,0.25){\(d\)}
\uput{2.5pt}[135](2.25,1.25){\(a\)}
\uput{2.5pt}[135](2.25,0.75){\(b\)}
}
\end{footnotesize}
\only<4->{\psset{linecolor=black}
\psline(2.25,1.25)(2.25,0.75)
\psline(2.25,0.75)(2.,0.25)
\psline(2.25,0.75)(2.5,0.25)
\psdots(2.25,0.75)
\psdots(2.25,1.25)
\psdots(2.,0.25)
\psdots(2.5,0.25)
}
\only<5->{{\psset{linestyle=dotted}
\psline(2,.25)(2.5,.25)
}}
\only<6->{
\psline{->}(3.,.75)(3.25,.75)
}
\begin{footnotesize}
\only<6->{\psset{labelsep=0pt}
\uput{2.5pt}[135](3.75,0.5){\(c\)}
\uput{2.5pt}[135](3.75,0.0){\(d\)}
\uput{2.5pt}[135](3.75,1.5){\(a\)}
\uput{2.5pt}[135](3.75,1.0){\(b\)}
}
\end{footnotesize}
\only<6->{\psset{linecolor=black}
\psline(3.75,1.5)(3.75,1.0)
\psline(3.75,1.0)(3.75,0.5)
\psline(3.75,1.0)(3.75,0.0)
\psdots(3.75,1.0)
\psdots(3.75,1.5)
\psdots(3.75,0.5)
\psdots(3.75,0.0)
}
}
\end{pspicture}
}\hspace{3mm}\parbox[b][3cm][c]{.55\textwidth}{\centering
\only<2>{A partially ordered set for which we want to find a linear extension.}
\only<3>{We query the oracle with \(a \ask{\le} b\).}
\only<4>{The answer is ``yes''.}
\only<5>{We query the oracle with \(d \ask{\le} c\).}
\only<6>{The answer is ``no''.}}
\end{ex}
\end{frame}

\begin{frame}\frametitle{\insertsection}\justifying
\begin{thm}[(\textsc{ITLB} of \textsc{SUPI})]
Given a poset \(\P\), computing a linear extension of \(\P\) requires at
least \(\log e(\P)\) queries.
\end{thm}\pause
\proof
A decision tree that computes a linear extension of \(\P\) has \(e(\P)\)
leaves. The height of this tree is at least \(\log e(\P)\).
\endproof\pause

\parbox[b][][c]{1\textwidth}{\centering
\psset{unit=.75cm}
\begin{pspicture}(-2.,-3.5)(5.5,3.5)
\only<3->{\psset{fillstyle=solid,fillcolor=darkgray!3!white,linecolor=darkgray,linewidth=1.35pt}
\rput{-0.916}(1.646,-0.026){\psellipse(0,0)(3.526,3.118)}}
\only<4>{\psset{fillstyle=solid,fillcolor=MediumOrchid2!18!white,linecolor=MediumOrchid2,linewidth=1.35pt}
\rput{40.892}(2.029,-1.02){\psellipse(0,0)(1.622,1.332)}}
\begin{small}
\only<3->{\uput[45](3.726,2.418){\color{darkgray}\(n!\)}}
\only<4>{\uput[45](3.448,-0.275){\color{MediumOrchid2}\(e(\P\))}}
\end{small}
\end{pspicture}}
\end{frame}

\section{\(k\)-linear Degeneracy Testing}
\begin{frame}\frametitle{\insertsection}\justifying
\begin{probl}[\(k\)-linear Degeneracy Testing (\(k\)-\textsf{LDT})]
Given a set \(\S \subset \R\), \(\card{\S} = n\), and \(k+1\) real
coefficients \(\alpha_0, \alpha_1, \ldots, \alpha_k \in \R\), decide whether
there exists a \(k\)-tuple
\((s_1, \ldots, s_k)\), \(s_i \in \S\), such that
\(\alpha_0 + \sum_{i=1}^{k} \alpha_i s_i = 0\).
\end{probl}\pause
We consider a model where a comparison can be made
between more than two elements. A ``comparison'' between three elements
\(a\), \(b\), and \(c\) could be
for example \(2a - 3b + c \ask{\le} 0\). Decisions based on the result of such
a comparison depend on the sign of the expression \(2a - 3b + c\), that is, whether
this expression is less than, equal to or greater than \(0\).
In this model, we are only allowed to query for linear combinations of
input elements. This generic model is the \emph{linear decision tree
model}.
\end{frame}

\section{Application of Meiser's Algorithm}
\begin{frame}\frametitle{\insertsection}\justifying
\end{frame}


\end{document}
