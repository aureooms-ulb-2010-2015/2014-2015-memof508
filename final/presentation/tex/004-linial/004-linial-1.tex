\section{Efficient Implementation of Linial's Algorithm}
\begin{frame}\frametitle{\insertsection}\justifying
\begin{ebox}{Merging under Partial Information}
\parbox[b][4cm][c]{.5\textwidth}{\centering
\psset{unit=.05cm}
\begin{pspicture}(-15,53)(65,-11)
{\psset{linewidth=1.0pt}
\only<1-3>{\psset{linecolor=black}
\uput{0}[0](-17,52){\textbf{input} \(\P = (\S = \enum{x_1,\ldots,x_N},<_{\P})\)}
}
\only<2->{\psset{fillstyle=solid,fillcolor=darkgray!3!white,linecolor=darkgray,linewidth=1.35pt}
\uput{0}[90](5,39){\(\S\)}
\psellipse(5,10)(15,27)
}
\only<4->{
\psset{fillstyle=solid,fillcolor=MediumOrchid2!18!white,linecolor=MediumOrchid2,linewidth=1.35pt}
\uput{0}[90](0,26){\(\A\)}
\psellipse(0,10)(5,10)

\psset{fillstyle=solid,fillcolor=DeepSkyBlue2!18!white,linecolor=DeepSkyBlue2,linewidth=1.35pt}
\uput{0}[90](10,26){\(\B\)}
\psellipse(10,10)(5,15)
}
\only<4>{
\uput{0}[0](-17,52){\textbf{input} \(\P = (\S = \A \cup \B, <_{\P})\)}
}
\only<5->{
\uput{0}[0](-17,52){\textbf{input} \(\P = (\S = \enum{a_1<_{\P}\cdots<_{\P}a_m} \cup \enum{b_1<_{\P}\cdots<_{\P}b_n},<_{\P})\)}
}
\only<2->{\psset{linecolor=black}
\psdots(0,5)
\psdots(0,15)
\psdots(10,0)
\psdots(10,10)
\psdots(10,20)
}
\only<3->{
\psset{linecolor=black}
\psline(0,5)(0,15)
\psline(0,5)(10,0)
\psline(0,15)(10,10)
\psline(0,5)(10,20)
\psline(10,0)(10,10)
\psline(10,10)(10,20)
\uput{0}[90](5,-9){\(<_{\P}\)}
}
\only<6->{
\psset{linecolor=black}
\psline{->}(30,10)(40,10)
\psline(65,-10)(65,30)
\psdots(65,-10)
\psdots(65,0)
\psdots(65,10)
\psdots(65,20)
\psdots(65,30)
\uput{5}[-90](65,-10){\(<\)}
}
}
\end{pspicture}
}\hspace{3mm}
\parbox[b][4cm][t]{.32\textwidth}{
\psset{unit=.05cm}
\begin{pspicture}(-10,48)(50,0)
{\psset{linewidth=1.0pt}
\only<7->{\psset{linecolor=black}

%\psdots(50,90) root
% left subtree
\psline(13,32)(5,25)
\psline(5,25)(0,20)
\psline(5,25)(10,20)
\psdots(0,20)
\psdots(10,20)
% right subtree
\psline(13,32)(23,22)
% right left subtree
\psline(23,22)(13,12)
\psline(13,12)(8,7)
\psline(13,12)(18,7)
\psdots(18,7)
% right left left subtree
\psline(8,7)(3,0)
\psline(8,7)(13,0)
\psdots(3,0)
\psdots(13,0)
% right right subtree
\psline(23,22)(33,12)
\psline(33,12)(26,2)
\psline(33,12)(40,2)
\psdots(26,2)
\psdots(40,2)

}
\only<8->{\psset{linecolor=black}
\psdots(2,-8)
\uput{0pt}[0](-7,-8){\(\#~~ = e(\P) \le n!\)}
}
\only<9->{\psset{linecolor=black}
\uput{0pt}[0](-9,-18){\(\textsc{ITLB} = \log e(\P)\)}
}
\only<10->{\psset{linecolor=black}
\psline{<->}(50,0)(50,32)
\uput{5pt}[0](50,16){\(h \ge \log e(\P)\)}
}
}
\end{pspicture}
}

\end{ebox}

\onslide<11->{%
\begin{thm}[(\citet*{linial:1984})]
Given a poset \(\P = (\enum{x_1,\ldots,x_N},<_{\P})\) covered by two chains
\(\A\) and \(\B\), we can always find a query \(x_i \ask{<} x_j\) with \(x_i
\in \A, x_j \in \B\) such that the probability that \(x_i < x_j\) lies in the
interval \([\sfrac{1}{3}, \sfrac{2}{3}]\).
\end{thm}%
}

\end{frame}
