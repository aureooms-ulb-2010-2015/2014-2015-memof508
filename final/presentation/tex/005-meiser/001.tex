\section{Application of Meiser's Algorithm}
\begin{frame}\frametitle{\insertsection}\justifying

\begin{ebox}{Point Location in an Arrangement of Hyperplanes}
\parbox[b][4cm][c]{.5\textwidth}{\centering
\psset{unit=.05cm}
\begin{pspicture}(0,61)(80,-3)
\only<3->{\psset{linewidth=1.0pt}
\pspolygon*[linecolor=MediumOrchid2!50!white](41,51.5)(70.5,41.75)(44,10)
}
\only<2->{\psset{linewidth=1.0pt}
\psdots(50,35)
\uput{2}[45](50,35){\(x\)}
}
\only<4->{\psset{linewidth=1.0pt}
\uput{0}[0](0,-5){\(\alpha_0 + \alpha_1 x_1 + \cdots + \alpha_n x_n = 0\)}
\psline[linewidth=1pt,linearc=5]{->}(74,-2)(76,2)(79,5)
\uput{1}[180](80,12){\(< 0\)}
\uput{1}[0](80,12){\(> 0\)}
}
{\psset{linewidth=1.0pt}
\psline(0,32)(50,64)
\psline(90,36)(0,64)
\psline(80,0)(81,64)
\psline(35,0)(90,64)
\psline(45,0)(40,64)
}
\end{pspicture}
}\hspace{3mm}
\parbox[b][4cm][t]{.32\textwidth}{
\psset{unit=.05cm}
\begin{pspicture}(-10,48)(50,0)
{\psset{linewidth=1.0pt}
\only<5->{\psset{linecolor=black}

%\psdots(50,90) root
% left subtree
\psline(13,32)(5,25)
\psline(5,25)(0,20)
\psline(5,25)(10,20)
\psdots(0,20)
\psdots(10,20)
% right subtree
\psline(13,32)(23,22)
% right left subtree
\psline(23,22)(13,12)
\psline(13,12)(8,7)
\psline(13,12)(18,7)
\psdots(18,7)
% right left left subtree
\psline(8,7)(3,0)
\psline(8,7)(13,0)
\psdots(3,0)
\psdots(13,0)
% right right subtree
\psline(23,22)(33,12)
\psline(33,12)(26,2)
\psline(33,12)(40,2)
\psdots(26,2)
\psdots(40,2)

}
\only<6->{\psset{linecolor=black}
\psdots(2,-8)
\uput{0pt}[0](-7,-8){\(\#~~ = \Phi_d(m) = \BigO{m^d}\)}
}
\only<7->{\psset{linecolor=black}
\uput{0pt}[0](-9,-18){\(\textsc{ITLB} = \BigO{d \log m}\)}
}
\only<8->{\psset{linecolor=black}
\psline{<->}(50,0)(50,32)
\uput{5pt}[0](50,16){\(h = \BigOmega{d \log m}\)}
}
}
\end{pspicture}
}


\end{ebox}

\begin{probl}[\(k\)-linear Degeneracy Testing (\(k\)-\textsf{LDT})]
Given a set \(\S \subset \R\), \(\card{\S} = n\), and \(k+1\) real
coefficients \(\alpha_0, \alpha_1, \ldots, \alpha_k \in \R\), decide whether
there exists a \(k\)-tuple
\((s_1, \ldots, s_k)\), \(s_i \in \S\), such that
\(\alpha_0 + \sum_{i=1}^{k} \alpha_i s_i = 0\).
\end{probl}
\end{frame}
