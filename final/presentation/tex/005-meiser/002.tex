\begin{frame}\frametitle{\insertsection}\justifying
\begin{algo}[\only<1-2>{(Idea of the algorithm)}\only<3->{(Meiser's algorithm)}]
\item[input] $x \in \R^n$, the point to be located.
\item[1.] Compute the position of \(x\) relative to a subset \(\H^{*}\)\only<3->{, {\color{Red3}\(\card{\H^{*}} = \BigO{n^2 \log^2 n}\)},} of \(\H\), finding the cell $\cell$ of \(\H^{*}\) containing $x$.
\only<3->{{\color{Red3}\item[2.] Build simplex $\simplex$ containing $x$ and inscribed in
$\cell$.}}
\only<1-2>{\item[2.] For any hyperplane $H_i$ not meeting \(\cell\), deduce $\pv_i(x)$
then discard the hyperplane.
\item[3.] Recurse on hyperplanes that are left.
}
\only<3->{\item[3.] For any hyperplane $H_i$ not meeting {\color{Red3}\(\simplex\)}, deduce $\pv_i(x)$
then discard the hyperplane.
\item[4.] Recurse on the hyperplanes that are left.
}
\end{algo}\pause
\begin{thm}[(\citet*{burgisser:1997})]
If we choose $\BigO{\enetsize}$ hyperplanes uniformly at
random from \(\H\) and denote this selection $\H^{*}$ and
if there is no hyperplane in $\H^{*}$ intersecting a given simplex, then, with
high probability, the number of hyperplanes of $\H$ intersecting the simplex
is less or equal to $\epsilon \card{\H}$.
\end{thm}
\end{frame}
