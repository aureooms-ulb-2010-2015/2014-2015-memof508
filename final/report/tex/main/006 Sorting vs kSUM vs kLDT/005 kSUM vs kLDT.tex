\section{\kSUM vs \kLDT}

The only important consequence of the difference between the definitions of
\kSUM and \kLDT is that all the permutations of the $k$-tuples need to be
considered in \kLDT. In \kSUM we have $\binom{n}{k}$ $k$-tuples to consider,
while in \kLDT we have $\binom{n}{k} k! = \frac{n!}{(n-k)!}$ $k$-tuples.

In \cite{gronlund:2014} a reduction of \kLDT to \threeSUM for $k$ odd is shown.
This reduction allows to match the same \BigO{n^{\sfrac{k}{2}} \sqrt{\log n}}
bound as the one established for the \kSUM problem in the same paper.

