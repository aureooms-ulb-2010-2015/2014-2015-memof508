\section{Strongly Polynomial Nonuniform Algorithm for the Subset Sum problem}

We now apply Meiser's Algorithm to the subset sum problem. In the subset sum
problem with input set $S \subset \R$ of cardinality $\card{S} = n$ we want to
find all subsets $S' \subset S$ whose elements sum up to zero, hence we have
to search through $2^n$ subsets of $S$. Reducing to the point location in an
arrangement of hyperplanes problem, we have now to locate input point $x \in
\R^n$ inside an arrangement of $2^n$ hyperplanes, i.e. all hyperplanes with
equation having combinations of $0$ or $1$ as coefficients. Meiser's Algorithm
solves this problem using only \BigO{n^4 \log^2 n} queries, i.e. $\poly$-time
in our decision tree model.

The subset sum problem is of interest here because it is an NP-complete problem
(\cite{karp:1972}), while our model of computation gives a polynomial time
algorithm. When the algorithm is described in \cite{burgisser:1997} it is not
assumed that the analysis is made in a different model than the classical
bit-model but rather than the size of the input $n$ is not part of the input
but rather known in advance, allowing one to construct an efficient data
structure for $\A(\H)$ for any instance of the problem with fixed $n$. This
kind of algorithme is refered as a \emph{nonuniform polynomial time algorithm}.
In this specific case it is also strongly polynomial since the computations
involved produce only real numbers whose size is bounded by the size of the
input.

\cite{burgisser:1997} and \cite{meiser:1993} give other examples of
NP-complete problems that can be solved with Meiser's algorithm.
