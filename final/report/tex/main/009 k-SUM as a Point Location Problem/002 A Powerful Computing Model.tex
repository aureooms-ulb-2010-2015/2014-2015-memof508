\section{A Powerful Computing Model}

Now that we have shown how to reduce $k$-SUM to a point location in an
arrangement of hyperplanes problem, we will see that with a sufficiently
powerful computing model it is possible to solve the subset sum problem in
polynomial time.

Since the beginning of this writing, we were only interested by the number of
questions we ask to an oracle, and we will continue to take the same point of
view here. We wonder how many times we need to effectively ask an oracle
whether a point lies above, on or under a certain hyperplane.

The algorithm we will describe determines the position vector $\pv(x) \in
\signset^{\binom{n}{k}} $ of input vertex $x = {x_1, \dots, x_n}$, effectively
solving the $k$-SUM problem using \BigO{n^3 \log^2 n} $n$-variate linear
queries.

For fixed $k$ we draw all hyperplanes having equation involving exactly $k$
coordinates of $\R^n$ with coefficient $1$, we have thus $\card{\H} =
\binom{n}{k}$. We can do this without involving the input vertex at all, this
costs us $0$ queries.

We give each hyperplane an orientation such that when point $x$ is below
hyperplane $H_i$, it is in the lower half-space defined by $H_i$ denoted
$\Him$. When point $x$ is above hyperplane $H_i$ it is in the upper
half-space defined by $H_i$ denoted $\Hip$. We define $\Hiz$ to be the
hyperplane $H_i$ itself. The $i^{th}$ component of the position vector is
$\pv(x)_i = \sigma$ if and only if $x \in \His$.

Our first idea to determine the components of $\pv(x)$ is to iteratively
select a random subset of size \BigO{\poly(n)} constitued of hyperplanes of the
arrangement $\Ht$ for which the relative position of $x$ has not yet been
determined and compute a position vector for this subset. Once we have located
vertex $x$ inside $\Ht$ we have effectively determined in
which cell of the arrangement $\Ht$ $x$ lies.

We are left with hyperplanes of the set $\H \setminus \Ht$. They can be
partitioned in two disjoint sets: set $\M$ of hyperplanes that meet the cell of
$\Ht$ containing $x$ and set $\O$ for the others. The relative position of $x$
to each hyperplane belonging to $\O$ can be deduced without the need to query
the oracle, the answer to the queries we would make can be computed by looking
at the structure of the arrangement itself. For the first group of hyperplanes
we just call the procedure again with $\H \gets \M$.

If we stop our explanation here we still have a worst case \BigO{\binom{n}{k}}
bound on the number of queries made. Hopefully, the next step of our
explanation will show that it is possible to lower this bound.

The problem we have with our current algorithm is that there is no guarantee
on the number of queries we avoid at each step.

We will use a result on $\epsilon$-nets to make our algorithm achieve an
upperbound of \BigO{n^3 \log^2 n} queries.

\begin{theorem}
Suppose we have a set $\H$ of $m$ hyperplanes in $\R^n$, if we choose
$\frac{n^2}{\epsilon} \log^2 \frac{n}{\epsilon}$ hyperplanes uniformly at
random from our $m$ hyperplanes and denote this selection $\H^{*}$ then for
any simplex intersecting at least $\epsilon$ hyperplanes of $\H$ there is a
high probability that at least one of them is contained in $\H^{*}$.
\end{theorem}

If we take the contraposition of this theorem we obtain the following
corollary,

\begin{corollary}
If there is no hyperplane $\in \H^{*}$ intersecting a given simplex then, with
high probability, the number of hyperplanes of $\H$ intersecting the simplex
is less than $\epsilon$.
\end{corollary}

We can use this corollary in the algorithm we were just describing earlier. We
now make two changes to the previous algorithm, we explicitely define the
cardinality of $\Ht$ and we include an additional step where we replace cell
$\cell$ with a simplex $\simplex$.

We choose $\card{\Ht} = \frac{n^2}{\epsilon} \log^2 \frac{n}{\epsilon}$,
remember $\Ht$ is the set of hyperplanes for which we will query the oracle in
the current step of the algorithm, and instead of directly discarding
hyperplanes not meeting the cell $\cell$ of the arrangement $\Ht$ containing $x$ we
refine the cell around point $x$ by computing a simplex $\simplex$ inscribed
to $\cell$ and containing $x$, and only after that discard hyperplanes that do
not meet the simplex.

By proceeding this way we have now the guarantee that we will keep at most a
constant fraction ($\epsilon$) of the hyperplanes and thus, by geometric
series, the total number of queries made to determine the enclosing cell of
each step is \BigO{\frac{n^2}{\epsilon} \log^2 \frac{n}{\epsilon}}. However, we
still need to explicit how we find a simplex $\simplex$ containing $x$ and
inscribed to $\cell$.

