\section{Reduction to a Point Location Problem}

We now describe a problem which can be used to model
various kind of set sum problems, including subset sum, \kSUM and \kLDT.

The problem is to locate a point \(x\) in the \(n\)-dimensional space \(\R^n\) relatively
to an arrangement of hyperplanes \(\A(\H)\). This arrangement divides the space
in cells of dimension \(0,\ldots,n-1\) and the goal will be to determine which
cell contains point \(x\). This amounts to determining for each
hyperplane \(H_i\) whether the point \(x\) lies above, on or below \(H_i\).

Formally, we define the point location problem in an arrangement of hyperplanes as
follows,

\begin{problem}
Given a point $x \in \R^n$ and an arrangement of hyperplanes $\H = \enum{ H_i
\st 1 \leq i \leq m }$ determine the position vector of $x$, $\pv(x) \in
\signset^n$ where $\pv_i(x) = \sigma, \sigma \in \signset$ iff $x \in
H_i^{\sigma}$. We define $H_i^{0}$, $H_i^{-}$ and $H_i^{+}$ to be
respectively $H_i$ itself, the half-space below $H_i$ and the half-space above
$H_i$.
\end{problem}

In the \kSUM problem, we are trying to find all subsets $S_i$ of the universe set
$U$ respecting the following constraints: $U$ is a finite subset of $\R$ and of
cardinality $n$, every subset $S_i$ of $U$ must contain exactly
$k$ elements of $U$ and the sum of those $k$ elements must be equal to $0$.
For a specific set $S$ we name those elements $s_1, \dots, s_k$.

We now describe a procedure to convert any instance of the \kSUM problem to an
instance of the point location in an arrangement of hyperplanes problem.

Let us note the bijection between $n$-tuples $(U_1, \dots, U_n)$ and points in
$\R^n$. By agreeing on an order of the elements of $U$ (e.g. the total order on
the real numbers) we can associate a unique $n$-tuple to each universe set $U$,
and thus a unique point $x$ in $\R^n$ for each $U$.

For each $S_i = \left\{s_1, \dots, s_k\right\}$ we need to find whether the sum $s_1
+ \dots + s_k$ is equal to zero. In our new representation of the problem this
boils down to find whether our point $x$ lies or not on the hyperplane of
equation $x_{i_1} + \dots + x_{i_k} = 0$. We thus draw every such hyperplane,
there are less than $n^k$ of them, and to solve the problem
we will have to find on which of them the point $x$ lies.

We just shown how to reduce \kSUM to a point location in an arrangement of
hyperplanes problem. In the next section, we will detail an algorithm due to
\citet*{meiser:1993} that can be used to solve \kSUM efficiently for any $k$.
There exists several methods to solve this problem. We will describe Meiser's
Algorithm~\cite{meiser:1993} since it is the first algorithm that solves this
point location problem in time polynomial in both $n$ and $\log m$, for fixed
$\H$ and $n$. This particular algorithm will give us the occasion to prove that
the decision tree complexity of subset sum is polynomial.

