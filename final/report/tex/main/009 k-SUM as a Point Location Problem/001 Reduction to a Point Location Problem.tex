\section{Reduction to a Point Location Problem}

In the $k$-SUM problem, we are trying to find all subsets $S_i$ of the universe set
$U$ respecting the following constraints: $U$ is a finite subset of $\R$ and of
cardinality $n$, every subset $S_i$ of $U$ must contain exactly
$k$ elements of $U$ and the sum of those $k$ elements must be equal to $0$.
For a specific set $S$ we name those elements $s_1, \dots, s_k$.

We now describe a procedure to convert any instance of the $k$-SUM problem to an
instance of the point location in an arrangement of hyperplanes problem.

Let us note the bijection between $n$-tuples $(U_1, \dots, U_n)$ and points in
$\R^n$. By agreeing on an order of the elements of $U$ (e.g. the total order on
the real numbers) we can associate a unique $n$-tuple to each universe set $U$,
and thus a unique point $x$ in $\R^n$ for each $U$.

For each $S_i = \left\{s_1, \dots, s_k\right\}$ we need to find whether the sum $s_1
+ \dots + s_k$ is equal to zero. In our new representation of the problem this
boils down to find whether our point $x$ lies or not on the hyperplane of
equation $x_{i_1} + \dots + x_{i_k} = 0$. We thus draw every such hyperplane,
there are less than $n^k$ of them, and to solve the problem
we will have to find on which of them the point $x$ lies.
