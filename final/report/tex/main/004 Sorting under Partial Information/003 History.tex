\section{Time Line}

The first appearance of this problem dates back to 1976. In
\cite{fredman:1976}, an algorithm making $\BigO{\log e(P) + 2n}$ queries was
featured. Fredman applies this algorithm to a problem he attributes to Elwyn
Berlekamp : sorting the set $X + Y = \enum{x + y \st x \in X, y \in Y}$. However
$\BigO{\log e(P)} \neq \BigO{n}$, so this algorithm was not matching the ITLB
for sub-linear $\log e(P)$. Moreover, the algorithm needs exponential time to
choose the right comparison to perform.

From that point, further research had to be made in order to get an algorithm
needing only $\BigO{\log e(P)}$ comparisons as well as an algorithm running in
polynomial time.

In \cite{kahn1984balancing}, it is showed that there always exists a query of
the form ``is $v_i \leq v_j$ ?'' such that the fraction of linear extensions in
which $v_i$ is smaller than $v_j$ lies in the interval $(\sfrac{3}{11},
\sfrac{8}{11})$. This is a relaxation of the well-known
\onethirdtwothird conjecture, a conjecture formulated
independently by M. Fredman, N. Linial, and R. Stanley, see
\cite{linial1984information}. Note that, a simpler proof yielding weaker bounds
was given by \cite{kahn1991balancing}. and better bounds were later given by
\cite{brightwell1995balancing}, and \cite{brightwell1999balanced}. In fact, the
only requirement for a practical use (without proving the conjecture) was to
prove that it was true for an interval $(q, 1-q)$, because then iteratively
choosing such a comparison yields an algorithm that performs $\BigO{\log e(P)}$
comparisons (precisely $\log_{q^{-1}} e(P)$). However, the algorithm proposed
by \cite{kahn1984balancing} does not find this \emph{optimal} query in
polynomial time.
