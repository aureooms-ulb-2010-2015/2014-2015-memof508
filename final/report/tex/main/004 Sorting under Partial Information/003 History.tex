\section{History}

The first appearance of this problem dates back to 1976. In
\citet*{fredman:1976}, an algorithm making $\BigO{\log e(P) + 2n}$ queries was
featured. Fredman applies this algorithm to a problem he attributes to Elwyn
Berlekamp : sorting the set $X + Y = \enum{x + y \st x \in X, y \in Y}$. However
$\BigO{\log e(P)} \neq \BigO{n}$, so this algorithm was not matching the ITLB
for sub-linear $\log e(P)$. Moreover, the algorithm needs exponential time to
choose the right comparison to perform.

From that point, further research had to be made in order to get an algorithm
needing only $\BigO{\log e(P)}$ comparisons as well as an algorithm running in
polynomial time.

In \citet*{kahn1984balancing}, it is showed that for any finite poset \(P\)
there always exists a query of the form \(a \ask{\leq} b\) with \(a,b \in P\)
such that the fraction of linear extensions in which $a$ is smaller than $b$
lies in the interval $(\sfrac{3}{11}, \sfrac{8}{11})$. This is a relaxation of
the well-known \onethirdtwothird conjecture, a conjecture formulated
independently by M. Fredman, N. Linial, and R. Stanley, see
\citet*{linial:1984}. Note that, a simpler proof yielding weaker
bounds was given by \citet*{kahn1991balancing} and better bounds were later
given by \citet*{brightwell1995balancing}, and \citet*{brightwell1999balanced}.
In fact, the only requirement for a practical use (without proving the
conjecture) was to prove that it was true for an interval $(q, 1-q)$, because
then iteratively choosing such a comparison yields an algorithm that performs
$\BigO{\log e(P)}$ comparisons (precisely $\log_{q^{-1}} e(P)$). However, the
algorithm proposed by \citet*{kahn1984balancing} does not find this
\emph{optimal} query in polynomial time.

For the sake of completenes, we hereunder state the \onethirdtwothird
conjecture,

\begin{conjecture}
In any finite poset $P$ that is not totally ordered, it is always possible to
find a pair $(a,b)$ of elements of $P$ such that the number of linear
extensions of $P$ in which $a$ comes before $b$ is between $\sfrac{1}{3}$ and
$\sfrac{2}{3}$ of the total number of linear extensions of $P$.
\end{conjecture}

The question of its corectness still remains and as far as we know the
$\sfrac{1}{3}$ bound is tight since we have proofs that the conjecture holds
for certain classes of posets. \citet*{linial:1984} for example gives a proof
for width-\(2\) posets. Also, their are some results due to
\citet*{peczarski:2006} for specific classes of partial orders verifying the
conjecture showing that the proportion of partial orders contained in those
classes approaches $1$ as $n$ grows. We can conclude that the proportion of
$n$-element partial orders obeing the \onethirdtwothird conjecture approaches
$1$.

Recent findings in the domain include \citet*{zaguia:2011} and
\citet*{peczarski:2008}.
