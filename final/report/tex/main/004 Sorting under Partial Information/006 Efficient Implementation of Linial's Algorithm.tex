\section{Efficient Implementation of Linial's Algorithm}

We present original developments on Linial's work~\cite{linial:1984}.
We explain
how to reduce the complexity of Linial's algorithm to \BigO{n^2} preprocessing
time and \BigO{n \log e(P)} running time.

As an anonymous referee pointed out in the peer review of
\citet*{cardinal:2013}, it is possible to implement Linial's algorithm
using dynamic programming instead of binomial coefficients and
determinants. We expose a recurrence relation that can be used to compute
\(e(\P)\) using dynamic programming.

We look at \(a_1\) and \(b_1\), the smallest elements of \(\A\) and \(\B\)
respectively. There are three ways \(a_1\) and \(b_1\) can be
related in the partial order \(\P\) we receive as input. Either we know \(a_1 <_{\P} b_1\)
or \(b_1 <_{\P} a_1\), or \(a_1\) and \(b_1\) are incomparable in \(\P\).
In the first two cases, we know that one of \(a_1,b_1\) must be the smallest
element of the set \(\A \cup \B\) when totally ordered. In these cases, \(e(\P) =
e(\P \setminus a_1)\) or \(e(\P) = e(\P \setminus b_1)\) for \(a_1 < b_1\) or
\(b_1 < a_1\) respectively. In the last case, if \(a_1\) and \(b_1\) are incomparable in
\(\P\), in the unknown total order we are trying to unveil one of \(a_1 < b_1\)
or \(b_1 < a_1\) holds true. Hence, \(e(\P) = e(\P \setminus a_1) + e(\P \setminus
b_1)\) in this case, since both outcomes for \(a \ask{<} b\) are possible.
Lastly, if \(\card{\A} = 0\) or \(\card{\B} = 0\) then \(e(\P) = 1\). We can thus
express \(e(\P)\) using the following recurrence relation
\begin{displaymath}
e(\P) =
\begin{dcases*}
1            & if \(\card{\A} = 0\) or \(\card{\B} = 0\)\\
e(\P \setminus {a_1}) & if \(a_1 <_{\P} b_1\)\\
e(\P \setminus {b_1}) & if \(b_1 <_{\P} a_1\)\\
e(\P \setminus {a_1}) + e(\P \setminus b_1) & if \(a_1\) and \(b_1\) are
incomparable in \(\P\),\\
\end{dcases*}
\end{displaymath}
which can be computed using dynamic programming. Note that posets \(\Q\) that
are incompatible with the original poset \(\P\),
are assigned a value \(e(\Q) = 0\).

If we compute \(e(\P)\) using this recurrence relation in a dynamic program,
we also get all values of \(e(\P(a_1 < b_r))\) and \(e(\P(b_1 < a_r))\) for
free. Without loss of generality we prove that this holds for all
values \(e(\P(a_1 < b_r)), 1 \le r \le \card{\B}\).
\begin{proof}
Let \(K\) be the tableau of our dynamic program, \ie
\begin{displaymath}
K = \enum{K_{i,j} = e(\P \cap (\enum{a_i,\ldots,a_{\card{\A}}} \cup
\enum{b_j,\ldots,b_{\card{\B}}}))}
\end{displaymath}
with \(1 \le i \le \card{\A}+1\) and \(1 \le j \le \card{\B}+1\). Since
\begin{displaymath}
K_{1,r+1} = e(\P \setminus \enum{b_1,\ldots,b_r})
\end{displaymath}
and
\begin{displaymath}
e(\P(a_1 < b_r)) = e(\P) - e(\P(b_r < a_1)) = e(\P) - e(\P \setminus \enum{b_1,\ldots,b_r})
\end{displaymath}
we have
\begin{displaymath}
e(\P(a_1 < b_r)) = K_{1,1} - K_{1,r+1}.\qedhere
\end{displaymath}
\end{proof}
Note that other values \(e(\P(a_i < b_j))\) for any pair \((a_i,b_j)\)
would be trickier to compute. The value of \(e(\P)\) and the number of linear
extensions of all poset extensions \(\P(a_1 < b_r)\) and \(\P(b_1 < a_r)\) can
thus be computed in \BigO{\card{\A}\card{\B} \log e(\P)} time using dynamic programming with a
\(\card{\A} \times \card{\B}\) memory tableau containing numbers with values up to \(\log
e(\P)\).

We can thus use the dynamic program to find a good query to perform instead of
computing determinants.
Moreover, we prove that we can update the information obtained via the dynamic program
with the answer to a query in \BigO{n} time.
\begin{proof}
Suppose that we have computed \(e(\P)\) and all \(e(\P(a_1 < b_r))\) and
\(e(\P(b_1 < a_r))\) using dynamic programming. According to these values and
thanks to Linial's proof of the \onethirdtwothird conjecture for width-\(2\)
posets, we find a good query \(a_1 \ask{<} b_r\) or \(b_1 \ask{<} a_r\) to perform.
Depending on the outcome of this query we update the tableau of the dynamic
program. We may assume the good query was of the type \(a_1 \ask{<} b_r\)
without loss of generality. If \(b_r < a_1\) then we do not have to update anything,
the chain \(\chain{b_1}{b_r}\) can simply be ignored. Otherwise, \(a_1 <
b_r\) and
unless \(r = 1\) we have to update at most \(\card{\A}\) cells of the dynamic program
tableau. The
cells to update are \(K_{1,1}\) through \(K_{1,r}\). The
new value to assign to \(K_{1,r}\) is \(K_{2,r}\) because if \(a_1 < b_r\)
then \(e(\P \setminus
\enum{b_1,\ldots,b_{r-1}}) = e(\P \setminus (\enum{b_1,\ldots,b_{r-1}} \cup
\enum{a_1})\). Since we changed \(K_{1,r}\) we also need to update all upstream
values \(K_{1,1}, \ldots, K_{1,r-1}\). This is the only case where we have to
update the tableau and the number of cells to update is \BigO{n}.
\end{proof}

We detail an implementation of the processing part of the algorithm. For the sake of simplicity,
we consider a recursive implementation. This implementation outputs a linear
extension compatible with the unknown total order \(<\). The algorithm receives as
input the poset \(\P\) covered by two chains \(\A =
(\chain{a_1}{a_{\card{\A}}})\) and \(\B = (\chain{b_1}{b_{\card{\B}}})\), the oracle
\(\ask{<}\) and a precomputed \(\card{\A} \times \card{\B}\) tableau \(K\).
\begin{algorithm}[Efficient implementation of Linial's algorithm]
\item[1.] If \(\card{\A} = 0 \text{ and } \card{\B} = 0\) the algorithm stops.
\item[2.1.] If \(\card{\B} = 0\) or \(a_1 <_{\P} b_1\) swap\footnote{Several times during the algorithm we use an instruction that swaps the
roles of \(\A\) and \(\B\). The intent is to avoid duplication of
instructions that would bloat the description. A side effect of this swap
instruction is that it exchanges
the roles of rows and columns in the tableau \(K\).}
the roles of \(\A\) and \(\B\).
\item[2.2.] If \(\card{\A} = 0\) or \(b_1 <_{\P} a_1\) output
\(b_1\), run the
algorithm with \(\B \gets \B \setminus \enum{b_1}\), then stop.
\item[3.1.] If \(\frac{e(\P(a_1 < b_1))}{e(\P)} > \sfrac{2}{3}\) swap the
roles of \(\A\) and \(\B\).
\item[3.2.] If \(\frac{e(\P(a_1 < b_1))}{e(\P)} < \sfrac{1}{3}\),
\item[3.2.1.] Compute \(r\) such that \(\sfrac{1}{3} \le \frac{e(\P(a_1 < b_r))}{e(\P)} \le \sfrac{2}{3}\).
\item[3.2.2.] Query the oracle\footnote{The only steps where we query the
oracle are \step{3.2.2} and
\step{3.3.1}.\label{footnote:linial:query}} with \(a_1 \ask{<} b_r\).
\item[3.2.3.] If \(b_r < a_1\) output
\(b_1,\ldots,b_r\), run the algorithm with \(\B \gets \B
\setminus \enum{b_1,\ldots,b_r}\), then stop.
\item[3.2.4.] Otherwise,
\item[3.2.4.1.] Update \(\P \gets \P( a_1 < b_r )\).
\item[3.2.4.2.] Update \(K_{1,j} \gets K_{2,j}\) for \(1 \le j \le r\).
\item[3.2.4.3.] Run the algorithm using the updated \(\P\) and \(K\), then stop.
\item[3.3.] Otherwise,
\item[3.3.1] Query the oracle\footnote{See~\cref{footnote:linial:query}.} with \(a_1 \ask{<} b_1\).
\item[3.3.2] If \(a_1 < b_1\) swap the roles of \(\A\) and \(\B\).
\item[3.3.3] Output \(b_1\), run the algorithm with \(\B \gets \B \setminus
\enum{b_1}\), then stop.
\end{algorithm}


As a final remark, it is possible to reduce the computational
complexity of both the construction and the update steps of the dynamic program. If
we limit ourselves to storing \emph{limited precision integers} and perform
\emph{limited precision arithmetic} when adding those integers, then we can
shave off the \(\log e(\P)\) factor in both cases. Hence, if we divide the
execution of the algorithm into a preprocessing phase and a processing phase,
the complexity of the preprocessing phase is \BigO{\card{\A}\card{\B}} while doing no
comparisons, and the complexity of the processing phase is \BigO{N \log e(\P)},
where \(N = \max \enum{\card{\A},\card{\B}}\). The algorithm uses at most
\(\log(\frac{1+\sqrt{5}}{2})^{-1} \log e(\P) \approx 1.44~\ITLB\)
queries as shown by \citet*{linial:1984}.
