\section{Orders}

This chapter talks about mathematical orders because they provide a standard
way to think about our intuitive notion of order.

Orders carry information about the set of objects we manipulate. An order is a
way to abstract the information that can be extracted from pairs of elements.

More formally we can look at an order by following the definition of a binary
relation, an ordered triple $(X, Y, G)$ where $X$ is the domain, $Y$ the
codomain and $G$ the graph of the relation.
Then we can say that elements that are contained in a binary relation $R$ are
of the type $x \in X~\text{is}~R\text{-related to}~y \in Y \iff (x, y) \in G$.

This abstraction is useful for our special cases, sorting, merging and, in the
next chapter, sorting under partial information, because the only information
we are interested in is the information concealed in some total order over a
specific finite set.

Note that here we will usually consider \(S = X = Y\).
