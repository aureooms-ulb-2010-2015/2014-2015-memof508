\section{Special Formations}

\subsection{Chains and Antichains}

In this subsection we describe two special structures that can be found in
posets. Those two structures are \concept{chains} and \concept{antichains}.
They convey the intuitive idea of respectively \emph{a sorted sequence of
elements} and \emph{a bag of incomparable elements}.

The words \concept{chain} and \concept{antichain} are not the only one used.
Indeed, they to have many synonyms; a \concept{chain} can also be called
\concept{totally ordered set} or \concept{linearly ordered set} and an
\concept{antichain} is also called \concept{Sperner family} or
\concept{clutter}.


Formally, a \concept{chain} is a poset in which any two elements are
comparable. Thus in \sref{ex:poset:def:a}, the poset $\bm{n}$ and all of its
subposets are chains. A subset $C$ of a poset $P$ is called a chain if $C$ is a
chain when regarded as a subposet of $P$.

Similarly, we define an \concept{antichain} as a subset $A$ of a poset $P$ such
that any two distinct elements of $A$ are incomparable.

\subsection{Maximal and Saturated Chains}

We talk about a maximal chain $C$ of $P$ if it is not contained in a larger
chain of $P$. For example in \sref{ex:poset:def:a}, $[1, n] = \bm{n}$ is the
only subposet of $\bm{n}$ that is maximal.

We call a chain $C$ of $P$ saturated (or unrefinable) if there does not exist
$y \in P - C$ that we could insert in chain $C$. If $x < y < z$ for some $x, y
\in C$ and $C \cup \left\{{y}\right\}$ is a chain then we can insert $y$ in $C$
and $C$ is not saturated. In \sref{ex:poset:def:a}, the poset $\bm{n}$ and all
of its intervals are saturated chains.

Maximal chains are thus saturated, however the converse is not true. In a
locally finite poset, a chain $x_0 < x_1 < \cdots < x_n$ is saturated if and
only if $x_{i-1}$ is covered by $x_i$ for $1 \le i \le n$.


\subsection{Lengths and Ranks}

Chains can be thought of as ordered sequences, and this notion of sequence make
us want to introduce the concept of length of a chain. We define the length
$\ell(C)$ of a finite chain by $\ell(C) = \#C - 1$. Using this definition, we
express the length (or rank) of a finite poset $P$ by the following formula

$$\ell(P) := max\left\{{\ell(C) : C ~\text{is a chain of}~ P}\right\}.$$

The length of an interval $[x, y]$ is denoted $\ell(s, t)$. In the case where
every maximal chain of $P$ has the same length $n$ we say that $P$ is graded of
rank $n$. In this case there is a unique rank function $\rho : P \to \left\{{0,
1, \cdots , n}\right\}$ such that $\rho(x) = 0$ if $x$ is a minimal element of
$P$ ($x \le y$ or $y \nleq x$ for all $y \in P$), and $\rho(t) = \rho(s) + 1$
if $t \gtrdot s$ in $P$. If $s \le t$ then we also write $\rho(s, t) = \rho(t)
- \rho(s) = \ell(s, t)$. If $\rho(s) = i$, then we say that $s$ has rank $i$.
This means that we can associate a rank to every element in the poset $P$.


\subsection{Order Ideals and Dual Order Ideals}

Here we look at structures that can be intuitively understood as generated by
selecting some subset $I'$ of elements of a poset $P$ and then extend this
selection by iteratively including all elements $y \in P$ such that they all
are either above an element of $I'$ or below an element of $I'$. These
structures are respectively called \concept{order ideals} and \concept{dual
order ideals}.

Again, many other words exists to refer to thoses structures. Synonyms for
\concept{order ideal} are \concept{semi-ideal}, \concept{down-set} and
\concept{decreasing subset}. Synonyms for \concept{dual order ideal} are
\concept{up-set}, \concept{increasing subset} and \concept{filter}.

Formally we have the two following definitions,

\begin{quotation}
An \concept{order ideal} of $P$ is a subset $I$ of $P$ such that if $x \in
I$ and $y \le x$, then $y \in I$.
\end{quotation}

\begin{quotation}
A \concept{dual order ideal} is a subset $I$ of $P$ such that if $x \in I$
and $y \ge x$, then $y \in I$.
\end{quotation}

Note that, when $P$ is finite, there is a one-to-one correspondence between
antichains $A$ of $P$ and order ideals $I$. Namely, $A$ is the set of maximal
elements of $I$, while

\begin{equation}
\label{eq:poset:a}
I = \left\{{x \in P : x \le z ~\text{for some z}~ \in A}\right\}.
\end{equation}

The set of all order ideals of $P$, ordered by inclusion, forms a poset denoted
$J(P)$. If $I$ and $A$ are related as in \ref{eq:poset:a}, then we say that $A$
\concept{generates} $I$. If $A = \enum{z_1 , \ldots , z_k}$, then we
write $I = \langle z_1 , \ldots , z_k \rangle$ for the order ideal generated by
$A$. The order ideal $\langle z \rangle$ is the \concept{principal order ideal}
generated by $z$, denoted $\Lambda_z$. Similarly $V_z$ denotes the principal
dual order ideal generated by $z$, that is, $V_z = \enum{x \in P \st x \ge
z}$.

