\section{Reduction to a Point Location Problem}

We now describe a problem\footnote{%
Note that we already used some of the ideas of this section several times in
the previous chapters.%
}
which can be used to model
various kind of set sum problems, including subset sum, \kSUM, and \kLDT.

Given a set of hyperplanes \(\H\) and a point \(x\) in the \(n\)-dimensional
space \(\R^n\), the problem is to locate \(x\) relatively
to the arrangement of hyperplanes \(\A(\H)\). This arrangement divides the space
into cells of dimension \(0,\ldots,n-1\) and the goal is to determine which
cell contains the point \(x\). This amounts to determining for each
hyperplane \(H_i\) whether the point \(x\) lies above, on, or below \(H_i\).
Formally, we define the point location problem in an arrangement of hyperplanes as
follows
\begin{problem}[Point location problem in an arrangement of hyperplanes]
Given a point $x \in \R^n$ and an arrangement of hyperplanes $\H = \enum{ H_i
\st 1 \leq i \leq m }$ determine the position vector of $x$, $\pv(x) \in
\signset^n$ where $\pv_i(x) = \sigma, \sigma \in \signset$ iff $x \in
H_i^{\sigma}$, where we define $H_i^{0}$, $H_i^{-}$ and $H_i^{+}$ to be
respectively $H_i$ itself, the half-space below $H_i$ and the half-space above
$H_i$.
\end{problem}
In the \kSUM problem, we want to decide whether there exists a subset \(\S\) of the
universe set \(\U = \enum{u_1,\ldots,u_n} \subset \R\) such that
$\S$
contains exactly $k$ elements of $\U$ and the sum of those $k$
elements is equal to $0$. For a given such subset $\S$ we name those elements
$s_1, \dots, s_k$.

We give a procedure to convert any instance of the \kSUM problem into an
instance of the point location problem in an arrangement of hyperplanes.
By agreeing on an order of the elements of $\U$ (e.g. the total order on
the real numbers) we can associate a unique $n$-tuple to each universe set $\U$,
and thus a unique point $x$ in $\R^n$ for each $\U$.
For each $\S_i = \left\{s_1, \dots, s_k\right\}$ we need to find whether the sum $s_1
+ \dots + s_k$ is equal to zero. In our new representation of the problem this
boils down to deciding whether our point $x$ lies on the hyperplane of
equation $x_{i_1} + \dots + x_{i_k} = 0$. There are less than $n^k$ such
hyperplanes, and to solve the problem
we have to decide whether \(x\) lies on one of them.

We just showed how to reduce \kSUM to a
point location problem in an arrangement of hyperplanes. In the next section,
we detail a location algorithm due to \citet*{meiser:1993} that can be used to
solve \kSUM efficiently for any $k$. There exist several methods to solve
this problem. We describe Meiser's Algorithm~\cite{meiser:1993} since it
is the first algorithm that solves this point location problem in time
polynomial in both $n$ and $\log m$. This particular
algorithm gives us the opportunity to prove that the decision tree complexity
of subset sum is polynomial.

