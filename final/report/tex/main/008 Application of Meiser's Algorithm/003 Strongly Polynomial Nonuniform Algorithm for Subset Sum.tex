\section{Strongly Polynomial Nonuniform Algorithm for Subset Sum}

We now apply Meiser's algorithm to the subset sum problem. In the subset sum
problem with input set $\S \subset \R$ of cardinality $\card{\S} = n$ we want to
find all subsets $\S' \subseteq \S$ whose elements sum up to zero. Hence, we have
to search through $2^n$ subsets of $\S$. Reducing to the point location problem
in an arrangement of hyperplanes, we have now to locate input point $x \in
\R^n$ inside an arrangement of $2^n$ hyperplanes, i.e. all hyperplanes with
equation having combinations of $0$ or $1$ as coefficients. Meiser's Algorithm
solves this problem using only \BigO{n^4 \log^2 n} queries, i.e. $\poly$-time
in our decision tree model.

The subset sum problem is of interest here because it is an NP-complete problem
\cite{karp:1972}, while our model of computation gives a polynomial time
algorithm. When the algorithm is described in \cite{burgisser:1997} it is not
assumed that the analysis is made in a different model than the classical
bit-model but rather that the dimension of $\R^n$, $n$, is not part of the input
but known in advance, allowing one to construct an efficient data structure for
$\arrangement(\H)$ for any instance of the problem with fixed $n$. This kind of
algorithm is referred to as a \emph{nonuniform polynomial time algorithm}.
This is a consequence of the nonuniformity of the model we use as we explained
in \ref{tree:sorting:nonuniformity}. In our case, the algorithm is
also strongly polynomial since the computations involved produce only real
numbers whose size is bounded by the size of the input.

\citet*{meiser:1993,burgisser:1997} give other examples of
NP-complete problems that can be solved with Meiser's algorithm.
