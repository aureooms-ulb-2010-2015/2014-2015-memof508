
% \section{Pancake sorting}


% Pancake sorting is the colloquial term for the mathematical problem of
% sorting a disordered stack of pancakes in order of size when a spatula can be
% inserted at any point in the stack and used to flip all pancakes above it.
% ``Pancake number'' refers to the maximum number of flips required for a given
% number of pancakes. In this form, the problem was first discussed by American
% geometer Jacob E. Goodman. It is a variation of the sorting problem in which
% the only allowed operation is to reverse the elements of some prefix of the
% sequence. Unlike a traditional sorting algorithm, which attempts to sort with
% the fewest comparisons possible, the goal is to sort the sequence in as few
% reversals as possible. A variant of the problem is concerned with burnt
% pancakes, where each pancake has a burnt side and all pancakes must, in
% addition, end up with the burnt side on bottom.


% References : \cite{cibulka2011average}, \cite{cohen1995problem}.
