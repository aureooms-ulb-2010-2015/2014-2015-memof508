\section{The Decision Tree Model}
\label{tree:sorting:model}

An important remark to begin with is that we are generally not interested
into the \emph{machine world} time-complexity of a problem. As always, we have to
define a model of computation that tells us which operations are allowed, which
are not and what are the costs for each of them.

From here on till the end we will work in the \emph{decision tree model}:
we are allowed to ask questions to an oracle \(\le\) that are answered
``yes'' or ``no''. Hence, each answer gives us at most
one additional bit of information.
Each question asked to the oracle costs us a single unit.
Every other operation can be carried out for free.

Indeed, solving a problem in
our model amounts to retrieve a certain quantity of information. We
need to find ways to organize the retrieved information and exploit already
known information. If done efficiently, we will solve our problems without
having to ask too many questions to our oracle.

The goal of each of our analyses is to either show that at least a certain
number of questions are required to be asked to solve a given problem, or
provide an algorithm solving this problem using at most a certain number of
questions. Those two kinds of results are called lower bounds and upper bounds
respectively. A lower bound or an upper bound is in general expressed as
a function of the input size.
