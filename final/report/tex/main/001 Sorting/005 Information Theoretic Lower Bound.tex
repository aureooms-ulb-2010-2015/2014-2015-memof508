\section{Information Theoretic Lower Bound}
\label{tree:sorting:ITLB}

We now give the Information Theoretic Lower Bound for the sorting problem. The
Information Theoretic Lower Bound is defined as the minimal depth of any
decision tree solving the problem. Generally speaking, the value of this lower
bound will be the logarithm in base \(2\) of the number of feasible solutions.

\nb{Throughout this document, \(\log N\) denotes the binary logarithm of \(N\).}

\begin{theorem}
The Information Theoretic Lower Bound (or ITLB) for the sorting problem is
\BigOmega{\log(n!)}. This means that, for any algorithm you may think of,
there will always be at least an instance of the problem that will force your
algorithm to ask \BigOmega{\log(n!)} questions to the oracle.
\end{theorem}

\begin{proof}
A sequence $s$ of length $n$ has $n!$ permutations. The decision tree has thus
$n!$ leaves, hence the minimal height of the tree is $\log(n!)$.
\end{proof}

This lower bound will be our goal to attain in terms of number of comparison
operations when designing an efficient algorithm solving our problem. The
bound only means that we cannot do better than \(\log(n!)\) comparisons but it
doesn't necessarily mean that there exists an algorithm or a decision tree
achieving this bound in terms of complexity.
