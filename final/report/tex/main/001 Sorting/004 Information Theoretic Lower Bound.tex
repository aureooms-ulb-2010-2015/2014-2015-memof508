\section{Information Theoretic Lower Bound}
\label{tree:sorting:ITLB}

\begin{theorem}
The Information Theoretic Lower Bound (or ITLB) for the sorting problem is
\BigOmega{\log(n!)}\footnote{Throughout this document, $\log x$ denotes the
binary logarithm of $x$.}. This means that, for any algorithm you may think of,
there will always be at least an instance of the problem that will force your
algorithm to ask \BigOmega{\log(n!)} questions to the oracle.
\end{theorem}

\begin{proof}
A sequence $s$ of length $n$ has $n!$ permutations. The decision tree has thus
$n!$ leaves, hence the minimal height of the tree is $\log(n!)$.
\end{proof}

\nb{We call this lower bound the information theoretic lower bound (ITLB)
because it is the logarithm of the number of feasible solutions.}
