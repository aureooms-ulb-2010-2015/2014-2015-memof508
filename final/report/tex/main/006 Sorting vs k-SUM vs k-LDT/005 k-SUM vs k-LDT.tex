\section{$k$-SUM vs $k$-LDT}

The only important consequence of the difference between the definitions of
$k$-SUM and $k$-LDT is that all the permutations of the $k$-tuples need to be
considered in $k$-LDT. In $k$-SUM we have $\binom{n}{k}$ $k$-tuples to consider,
while in $k$-LDT we have $\binom{n}{k} k! = \frac{n!}{(n-k)!}$ $k$-tuples.

In \cite{gronlund:2014} a reduction of $k$-LDT to $3$-SUM for $k$ odd is shown.
This reduction allows to find the same \BigO{n^{\sfrac{3}{2}} \sqrt{\log n}}
bound as in the $3$-SUM problem.

