\section{Sorting is a $2$-LDT problem}

As we will see right now, sorting is a $2$-LDT problem. Let us define the
$2$-LDT problem,

\begin{problem}
Given a set $S \subset \R$, $\card{S} = n$ and real coefficients $\alpha_0,
\alpha_1, \alpha_2$, find all $(a,b) \in S^2$ such that
$\alpha_0 + \alpha_1 a + \alpha_2 b = 0$.
\end{problem}

In the sorting problem we need to determine whether $a < b$ for all $(a,b)$
pairs, this sums up to determine the sign of $a-b$. If we would like to rewrite
the sorting problem definition in the same fashion we would come up with the
following definition,

\begin{problem}
Given a set $S \subset \R$, $\card{S} = n$, find all $(a,b) \in S^2$ such that
$a - b = 0$.
\end{problem}

All sorting problem instances are thus instances of the $2$-LDT problem with
$\alpha_0 = 0, \alpha_1 = 1, \alpha_2 = -1$.

The intuition is correct even though $a \neq b$ implies $a-b \neq 0$ for all
$(a,b) \in S^2$, and so there can not be such pair $(a,b)$, but, the important
bit is that computing $a-b$ reveals the answer to the question $a<b$ as we
noticed earlier. An algorithm that determines the sign of $a-b$ for all $(a,b)$
pairs is thus effectively retrieving the information required to sort the set
$S$.

Sorting and $2$-LDT are thus very similar problems. While sorting $S$ amounts
to determine the sign of $a-b$ for all $(a,b)$ pairs, solving the $2$-LDT
problem on set $S$ amounts to determine the sign of $\alpha_0 + \alpha_1 a +
\alpha_2 b = 0$ for the same pairs.
