\section{Similar Complexities for \kSUM and \kLDT}

The only important consequence of the difference between the definitions of
\kSUM and \kLDT is that all the permutations of the $k$-tuples need to be taken
into account in \kLDT. In \kSUM we have $\binom{n}{k}$ $k$-tuples to consider,
while in \kLDT we have $\binom{n}{k} k! = \frac{n!}{(n-k)!}$ $k$-tuples.

However, in \ref{tree:3sum:kldt} we show that it is possible to translate
\kLDT instances to
\kSUM instances and obtain the same upper bounds as \kSUM for $k$-linear
decision trees. If $k \ge 4$ is even, we can translate a \kLDT instance of
size \(n\) to a balanced \twoSUM instance of size \(2n^{\sfrac{k}{2}}\)
and obtain a \BigO{n^{\sfrac{k}{2}} \log n} algorithm. In the other case, when
$k \ge 3$ is odd, a \kLDT instance of size \(n\) can be translated to an unbalanced \threeSUM instance, where
\(\card{\A} = \card{\B} = n^{\frac{k-1}{2}}\) and \(\card{\C} = n\), that can be solved in
\BigO{n^{\frac{(k+1)}{2}}} time.

We show later that \citet*{gronlund:2014} improve the upper bound for the
\threeSUM problem. The translation of \kLDT to \threeSUM when $k$ is
odd still applies
and allows to match the same \BigO{n^{\sfrac{k}{2}} \sqrt{\log n}}
bound as the one established for \threeSUM in the same paper.

