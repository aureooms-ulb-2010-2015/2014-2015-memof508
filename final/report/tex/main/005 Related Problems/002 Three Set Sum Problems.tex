
\section{Three Set Sum Problems}

The three problems given in this section are about finding subsets whose
elements sum up to a certain constant. The first one is the subset sum problem
which is NP-complete~\cite{karp:1972}. The two others, \kSUM and \kLDT are variants of the
first one where we fix the size $k$ of the subsets to search for. \kLDT is
different from \kSUM as it doesn't really compute a sum in the strict sense.
It computes the image of a $k$-variate linear function $\phi(s_1, \ldots, s_k)
= \alpha_0 + \sum_{i=1}^{k} \alpha_i s_i$, allowing other values than $0$ or
$1$ for the coefficients $\alpha_0, \alpha_1, \ldots, \alpha_k$. It is trivial
to see that \kSUM is contained in \kLDT. We will explain in the next sections
how these set sum problems relate to sorting problems. Below are the formal
definitions of the problem.

The subset sum problem is the following,

\begin{problem}
Given a set $\U \subset \R$, $\card{\U} = n$, find all
$\S \subseteq \U$, such that $\sum\limits_{s \in \S} s = 0$.
\end{problem}


The \kSUM problem is the following,

\begin{problem}
Given a set $\U \subset \R$, $\card{\U} = n$, find all
$\S \subseteq \U$, $\card{\S} = k$, such that $\sum\limits_{s
\in \S} s = 0$.
\end{problem}


The $k$-variate linear degeneracy testing problem, abbreviated \kLDT, is the
following,

\begin{problem}
Given a set $\U \subset \R$, $\card{\U} = n$, and $k+1$ real
coefficients $\alpha_0, \alpha_1, \ldots, \alpha_k \in \R$, find all $k$-tuples
$S = (s_1, \ldots, s_k)$, $s_i \in \U$, such that
$\alpha_0 + \sum_{i=1}^{k} \alpha_i s_i = 0$.
\end{problem}
