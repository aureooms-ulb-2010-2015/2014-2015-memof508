\section{Sorting \XY is a \fourLDT problem}
\label{tree:related:xy4ldt}

Similarly to how the sorting problem is a linear degeneracy testing problem,
Sorting \XY is a \fourLDT problem. Let us define \fourLDT,

\begin{problem}[\fourLDT]
Given a set $\S \subset \R$, $\card{\S} = n$, and real coefficients $\alpha_0,
\alpha_1, \alpha_2, \alpha_3, \alpha_4$, decide whether there exists a
\(4\)-tuple \((a,b,c,d) \in \S^4\) such that
$\alpha_0 + \alpha_1 a + \alpha_2 b + \alpha_3 c + \alpha_4 d = 0$.
\end{problem}

When sorting \XY we need to determine whether $a + b \le c + d$ for all
$(a,b,c,d) \in (X \times Y)^2$, this sums up to determining the sign of $(a+b) -
(c+d)$, i.e. asking questions of the kind $a + b - c - d \ask{\le} 0$. We can thus
reformulate sorting \XY the following way,

\begin{problem}[Sorting \XY as a \fourLDT problem]
Given a set $\S \subset \R$, $\card{\S} = n$, decide whether there exists a
\(4\)-tuple \((a,b,c,d) \in \S^4, a \neq c \lor b \neq d\) such
that $a + b - c - d = 0$.
\end{problem}

Again, we want to prove that this problem is equivalent to Sorting \XY. We now
state and prove a more general theorem than \ref{thm:related:sorting}.

\begin{theorem}[The search and decision versions of the point location problem are equivalent]\label{thm:related:sortingxy}
Given a point \(p = (p_1, \ldots, p_n)\) and an arrangement of hyperplanes
\(\H\) in \(\R^n\), deciding whether \(p\) lies on one of the hyperplanes of the
arrangement amounts to finding the cell of the arrangement containing \(p\).
\end{theorem}

\begin{proof}
We assume \(p\) does not lie on any of the hyperplanes without loss of generality.
Suppose we have a hyperplane \(H \in \H, H = a_1 x_1 + \cdots + a_n x_n = 0\) for which we
know that \(p \notin H\) and assume \(a_1 p_1 + \cdots + a_n p_n \le 0\)
without loss of generality.
Then at least one of the following propositions holds:
\begin{enumerate}
\item We directly made the queries \(a_1 p_1 + \cdots + a_n p_n \le 0\) and
\(-a_1 p_1 + \cdots + -a_n p_n \le 0\), revealing the sign of the expression
\(a_1 p_1 + \cdots + a_n p_n\).
\item There is a subset of the arrangement \(\enum{H_1,\ldots,H_k}\) with
\(H_i = a_{i_1} x_1 + \cdots + a_{i_n} x_n = 0\) such that \(H\) is a linear
combination of these hyperplanes with coefficients \(\beta_i \neq 0\) for which we
know that \(a_{i_1} p_1 + \cdots + a_{i_n} p_n \le 0\) if \(\beta_i > 0\) or
\(-a_{i_1} p_1 + \cdots + -a_{i_n} p_n \le 0\) if \(\beta_i < 0\), and at
least one of \(a_{i_1} p_1 + \cdots + a_{i_n} p_n \neq 0\).
\end{enumerate}
In all cases knowing \(a_1 p_1 + \cdots + a_n p_n \neq 0\) implies knowing the
sign of \(a_1 p_1 + \cdots + a_n p_n\) and thus the answer to the query \(a_1
p_1 + \cdots + a_n p_n \ask{\le} 0\). Knowing all these answers one finds the
cell of the arrangement containing \(p\).
\end{proof}

We can see that Sorting \XY is a special case of the point location problem in
an arrangement of hyperplanes by mapping each possible query $a + b \ask{\le} c
+ d$ to a hyperplane $a + b - c - d = 0$.

Sorting \XY instances are thus instances of \fourLDT
with $\alpha_0 = 0, \alpha_1 = \alpha_2 = 1, \alpha_3 = \alpha_4 = -1$ and
$\S = X \cup Y$.
