\section{\(k\)-linear Decision Trees}

From now on, we have been considering that, in the decision tree
model we use, the only kind of query that we were allowed to ask was of the
form \(a \ask{\le} b \) where \(a\) and \(b\) are elements of some input set.

It is now time to upgrade our model by widening the concept of element
comparison. In the following sections we will consider that a comparison can
be made between more than just two elements. A comparison between three
elements could be for example \(2a - 3b + c \ask{=} 0\). Decisions based on the
result of such a comparison will depend on the sign of the expression
\(2a - 3b + c\) \ie whether this sum is smaller than, equal to or greater than
\(0\).

In this model, we will only be allowed to query for linear combinations of
input elements and those linear combinations can involve at most \(k\) of those
elements. We will call this generic model the \(k\)-linear decision tree model.

As a concrete example, let us take the sorting \XY problem we just introduced.
The set \XY contains all pairwise sums of elements of \(X\) with elements of
\(Y\). Thus, if we want to sort \XY we could make queries of the form \((x_i +
y_j) \ask{\le} (x_{i'} + y_{j'})\) and this can be rewritten as \( x_i +
y_j - x_{i'} - y_{j'} \ask{=} 0\) to fit our new model. Solving the problem
this way would involve \(4\)-linear queries and hence we would be working in
the \(4\)-linear decision tree model. Of course, we take the conditional here
because sorting \XY is in no way forcing us to use \(k=4\).
