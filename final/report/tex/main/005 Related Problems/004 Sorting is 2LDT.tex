\section{Sorting is a \twoLDT problem}

As we show now, sorting is a \twoLDT problem. Let us define the
\twoLDT problem,

\begin{problem}
Given a set $\S \subset \R$, $\card{\S} = n$, and real coefficients $\alpha_0,
\alpha_1, \alpha_2$, find all $(a,b) \in \S^2$ such that
$\alpha_0 + \alpha_1 a + \alpha_2 b = 0$.
\end{problem}

In the sorting problem we need to determine whether $a < b$ for all $(a,b)$
pairs, this sums up to determine the sign of $a-b$. If we would like to rewrite
the sorting problem definition in the same fashion we would come up with the
following definition,

\begin{problem}
Given a set $\S \subset \R$, $\card{\S} = n$, find all $(a,b) \in \S^2$ such that
$a - b = 0$.
\end{problem}

Let us prove why this definition maps to the definition of the \twoLDT
problem.

\begin{theorem}
Deciding whether there exists a pair \((a,b)\) such that \(a - b = 0\)
amounts to finding signs of all \(a-b\).
\end{theorem}

\begin{proof}
We use an adversary argument. Imagine we do not know the sign for one of the
pairs \(s_i,s_j\), then in the case of the sorting problem we still have two
total orders that satisfy all the relations for which we have information. For
example the two total orders
\begin{displaymath}
s_{i_1} \le s_{i_2} \le \ldots \le s_i \le s_j \le \ldots \le s_{i_n}
\end{displaymath}
and
\begin{displaymath}
s_{i_1} \le s_{i_2} \le \ldots \le s_j \le s_i \le \ldots \le s_{i_n}
\end{displaymath}
differ only by the answer to \(s_i \ask{\le} s_j\).
Similarly, in the \twoLDT problem we cannot conclude that there is no pair
\(a,b\) such that \(a-b=0\) unless we have taken all of them into account.
\end{proof}

All sorting problem instances are thus instances of the \twoLDT problem with
$\alpha_0 = 0, \alpha_1 = 1, \alpha_2 = -1$.

The intuition is correct even though $a \neq b$ implies $a-b \neq 0$ for all
$(a,b) \in \S^2$, and so there cannot be such pair $(a,b)$, but, the important
bit is that computing $a-b$ reveals the answer to the question $a<b$ as we
noticed earlier. An algorithm that determines the sign of $a-b$ for all $(a,b)$
pairs is thus effectively retrieving the information required to sort the set
$\S$.

Sorting and \twoLDT are thus similar problems. While sorting $\S$ amounts
to determine the sign of $a-b$ for all $(a,b)$ pairs, solving the \twoLDT
problem on set $\S$ amounts to determine the sign of $\alpha_0 + \alpha_1 a +
\alpha_2 b = 0$ for the same pairs.
