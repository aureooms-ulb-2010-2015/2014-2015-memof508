\section{Sorting is a \twoLDT problem}

We show that Sorting is a \twoLDT problem. Let us define \twoLDT
\begin{problem}[\twoLDT]
Given a set $\S \subset \R$, $\card{\S} = n$, and real coefficients $\alpha_0,
\alpha_1, \alpha_2$, decide whether there exists a pair $(a,b) \in \S^2$ such that
$\alpha_0 + \alpha_1 a + \alpha_2 b = 0$.
\end{problem}

In the sorting problem we need to determine whether $a \le b$ for all $(a,b)$
pairs, this amounts to determining the sign of $a-b$. If we would like to rewrite
the sorting problem definition in the same fashion we would come up with the
following definition
\begin{problem}[Sorting as a \twoLDT problem]
Given a set $\S \subset \R$, $\card{\S} = n$, decide whether there exists a
pair $(a,b) \in \S^2, a \neq b$ such that $a - b = 0$.
\end{problem}

The intuition is correct even though $a \neq b$ implies $a-b \neq 0$ for all
pairs \((a,b)\), and so there cannot be such pair \((a,b)\), but, the important
bit is that computing $a-b$ reveals the answer to the question $a \ask{\le} b$.

Let us prove why this definition maps to the definition of the Sorting problem.
\begin{theorem}[Sorting is a \twoLDT problem]\label{thm:related:sorting}
Given a set \(\S \subset \R\), deciding whether there exists a pair \((a,b)
\in \S^2, a \neq b\) such that \(a - b = 0\) amounts to finding the answers of all \(a
\ask{\le} b\) queries.
\end{theorem}
\begin{proof}
We may assume there is no such pair without loss of generality.
Suppose that we have a pair \((a,b) \in \S^2, a \neq b\) for which we
know that \(a - b \neq 0\) and assume \(a \le b\) without loss of generality.
Then at least one of the following propositions holds:
\begin{enumerate}
\item We directly made the queries \(a - b
\ask{\le} 0\) and \(b - a \ask{\le} 0\), revealing the sign of the expression
\(a - b\).
\item There is a \(x \in \S\) such that \(a \le x \le b\) for which we
know that \(a - x \le 0\), \(x - b \le 0\), and at least one of \(a - x \neq
0\) or \(x - b \neq 0\).
\end{enumerate}
Since \(a - x \le 0 \land x - b \le 0 \iff a - x + x - b \le 0 \iff a - b \le
0\), in all cases knowing \(a - b \neq 0\) implies knowing the sign of \(a -
b\) and thus the answer to the query \(a \ask{\le} b\).
\end{proof}

All sorting problem instances are thus instances of \twoLDT with
$\alpha_0 = 0, \alpha_1 = 1, \alpha_2 = -1$.
