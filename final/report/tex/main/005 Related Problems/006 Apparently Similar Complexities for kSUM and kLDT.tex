\section{Apparently Similar Complexities for \kSUM and \kLDT}

The only important consequence of the difference between the definitions of
\kSUM and \kLDT is that all the permutations of the $k$-tuples need to be taken
into account in \kLDT. In \kSUM we have $\binom{n}{k}$ $k$-tuples to consider,
while in \kLDT we have $\binom{n}{k} k! = \frac{n!}{(n-k)!}$ $k$-tuples.

\emph{TODO: what follows is purely speculative.}

However, it is possible to reduce\footnote{see \ref{tree:3sum:kldt}} \kLDT to
\kSUM problems and obtain the same upper bounds as \kSUM for $k$-linear
decision trees. If $k \ge 4$ is even, one can reduce \kLDT to a \twoSUM problem
and obtain a \BigO{n^{\sfrac{k}{2}} \log n} algorithm. In the other case, when
$k \ge 3$ is odd, \kLDT is reducible to an unbalanced \threeSUM problem, where
$\card{\A} = \card{\B} \gg \card{\C}$, that can be solved in
\BigO{n^{\sfrac{(k+1)}{2}}} time.

We will see later that \citet*{gronlund:2014} improve the upper bound for the
\threeSUM problem. The standard reduction of \kLDT, $k$ is odd, still applies
and allows to match the same \BigO{n^{\sfrac{k}{2}} \sqrt{\log n}}
bound as the one established for \threeSUM in the same paper.

