\section{$s$-linear Decision Trees}

\citet*{ailon:2005} study $s$-linear decision trees to solve the \kSUM
problem when $s > k$. In particular, they give an additional proof for the
\BigOmega{n^{\ceil{\frac{k}{2}}}} lower bound of \citet{erickson:1999} and
establish new lower bounds for all $s > k$. The case $s = k+1$ leads to a
subquadratic lower bound for \threeSUM. However, at that point, we are still
missing an actual subquadratic algorithm for \threeSUM.

Note that the exact lower bound given by \citet*{erickson:1999} for \(s = k\) is
\BigOmega{(nk^{-k})^{\ceil{\frac{k}{2}}}} while the one given by
\citet*{ailon:2005} is \BigOmega{(nk^{-3})^{\ceil{\frac{k}{2}}}}. Their result
improves thus the lower bound when \(k\) is large.

The bound they prove for \(s > k\) is the following,
\begin{theorem}[\citet*{ailon:2005}]
For any instance of \kLDT, the tree depth is at least
\begin{displaymath}
\BigOmega{\group{nk^{-3}}^{\frac{2k-s}{2\ceil{\frac{s-k+1}{2}}} (1-\epsilon_k) }}
\end{displaymath}
where \(\epsilon_k > 0\) tends to \(0\) as \(k \to\infty\).
\end{theorem}

In the next section we talk about the first successful attempt at finding
a subquadratic algorithm for the \threeSUM problem.
