\section{$s$-linear Decision Trees}

\citet*{ailon:2005} study $s$-linear decision trees to solve the \kSUM
problem when $s > k$. In particular, they give an additional proof for the
\BigOmega{n^{\ceil{\frac{k}{2}}}} lower bound of \citet{erickson:1999} and
generalize the lower bound for the \(s\)-linear decision tree model when $s > k$.

Note that the exact lower bound given by \citet*{erickson:1999} for \(s = k\) is
\BigOmega{(nk^{-k})^{\ceil{\frac{k}{2}}}} while the one given by
\citet*{ailon:2005} is \BigOmega{(nk^{-3})^{\ceil{\frac{k}{2}}}}. Their result
improves therefore the lower bound for \(s = k\) when \(k\) is large.

The lower bound they prove for \(s > k\) is the following
\begin{theorem}[\citet*{ailon:2005}]
For any instance of \kLDT, the tree depth is at least
\begin{displaymath}
\Omega\mleft(\group{nk^{-3}}^{\frac{2k-s}{2\ceil{\frac{s-k+1}{2}}}
(1-\epsilon_k) }\mright),
\end{displaymath}
where \(\epsilon_k > 0\) tends to \(0\) as \(k \to\infty\).
\end{theorem}

We can interpret this result as a manifestation of the fact that allowing \(s >
k\) can lead to more powerful models of computation, as we will see in the next
section and in the last chapter.

However, this new lower bound breaks down when
\(k = n^{\sfrac{1}{3}}\) or \(s = \BigOmega{k}\) and the cases where \(r <
6\), \(s > r\) give trivial lower bounds. For example, in the case
of \threeSUM with \(s = k + 1\) we get a lower bound that is \BigO{n}. This is
worse than the general lower bound of \BigOmega{n \log n}.

In the next section we discuss the first successful attempt at finding
a subquadratic algorithm for the \threeSUM problem.
