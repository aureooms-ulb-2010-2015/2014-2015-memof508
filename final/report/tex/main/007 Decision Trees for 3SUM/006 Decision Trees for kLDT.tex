\section{Decision Trees for \kLDT}
\label{tree:3sum:kldt}

In this section we will first show a transformation to construct an instance
of \threeSUM from an instance of \kLDT for the case where \(k\) is odd. Next, we will
see that in the \(k\)-linear decision tree model one can solve \kLDT optimally
using the standard quadratic algorithm for \threeSUM. After that, we will give a
short argument to explain that the same kind of technique can be applied when
\(k\) is even. Finally, we illustrate how the decision tree for \threeSUM of
\citet*{gronlund:2014} gives a better lower bound, for odd \(k\), using
\((2k-2)\)-linear decision trees.

The goal of this section is to prove the following result,

\begin{theorem}
When $k \ge 3$ is odd, there is a $(2k-2)$-linear decision tree for \kLDT with
depth \BigO{n^{\sfrac{k}{2}} \sqrt{\log n}}.
\end{theorem}

The standard three-set version of the quadratic \threeSUM algorithm for input
sets \(\A\), \(\B\) and \(\C\) completes after \BigO{\card{\C} ( \card{\A} +
\card{\B} ) } steps. We will show how to solve any instance of a \kLDT problem
with \(k \ge 3\) odd in time \BigO{n^\frac{k+1}{2}}.

Given an input set \(\S\) of \(n\) real numbers, a set of \(k\) real
coefficient \( \enum{\alpha_1,\ldots,\alpha_k}\) and a real term \(\alpha_0\)
we construct the sets \(\A\), \(\B\) and \(\C\) as follows,

\begin{displaymath}
\A = \enum{\alpha_0 + \alpha_1 s_1 + \cdots + \alpha_{\frac{k-1}{2}} s_{\frac{k-1}{2}} \st s_i \in \S}
\end{displaymath}

\begin{displaymath}
\B = \enum{\alpha_{\frac{k+1}{2}} s_{\frac{k+1}{2}} + \cdots + \alpha_{k-1} s_{k-1} \st s_i \in \S}
\end{displaymath}

\begin{displaymath}
\C = \enum{\alpha_k s_k \st s_k \in \S}
\end{displaymath}

Each element of \(\A\) and \(\B\) has exactly \(\frac{k-1}{2}\) linear terms.
Elements of \(\A\) have an additional \(\alpha_0\) independent term. Looking
at all possible combinations of \(s_i\) with \(\alpha_j\) we know that
\(\card{\A} = \card{\B} = \BigO{n^{\frac{k-1}{2}}}\), whereas \(\card{\C} =
n\). As a remark, note that by simply omitting \(\alpha_j\) factors we obtain a
similar reduction from \kSUM.

If we use the quadratic \threeSUM algorithm to solve the instance we
constructed, it will make \(k\)-linear queries since summing an element of
\(\A\) with an element of \(\B\) and comparing this sum to an element of
\(\C\) will involve \(2 \frac{k-1}{2} + 1\) linear terms. The running time of
the \threeSUM algorithm on this input is \(\BigO{n \cdot 2 \cdot n^{\frac{k-1}{2}}} =
\BigO{n^{\frac{k+1}{2}}}\).

Note that to use this algorithm we need to construct sets \(\A,\B,\C\).
Moreover we have to provide sorted structures for \(\A\) and \(\B\). However
the complexity of this preprocessing is only \BigO{n^{\frac{k-1}{2}} \log n}.

For \(k \ge 4\) even one constructs only two sets \(\A\) and \(\B\) of
cardinality \(\card{\A} = \card{\B} = \BigO{n^{\frac{k}{2}}}\). By doing this
we transormed our instance of \kLDT to an instance of \twoSUM. \twoSUM can be
solved in \BigO{n^{\frac{k}{2}} \log n} with only \BigO{n^{\frac{k}{2}}}
comparisons using Lambert's algorithm \cite{lambert:1990}.

\citet*{gronlund:2014} explain that one can generalize their new decision tree
for \threeSUM from the one-set version to a three-set version. This version of
the decision tree has complexity

\begin{displaymath}
\BigO{g(\card{\A}+\card{\B}) + g^{-1}\card{\C}(\card{\A}\card{\B})\log g},
\end{displaymath}

hence if one chooses \(g = \sqrt{n \log n}\) one will obtain a
\BigO{n^{\frac{k}{2}} \sqrt{\log n}} decision tree for \kLDT with $k$ odd
which improves the best known lower bound.

Also, we can note that this way of doing makes use of \((2k-2)\)-linear queries
instead of the \(k\)-linear queries used in the standard \threeSUM
algorithm or in Lambert's algorithm. Indeed, since \(\card{\A} = \card{\B} =
\BigO{n^{\frac{k-1}{2}}}\), when we sort all \((\A_i - \A_i)\) and \((\B_i -
\B_i)\) we are making \(4 \frac{k-1}{2} = (2k-2)\)-linear queries.

Note that with both the classical algorithm and the algorithm of
\citet*{gronlund:2014}, the time required to build an instance of \twoSUM or
\threeSUM from an instance of \kLDT is smaller than the depth of the decision
trees used to solve the problem. One must also note that it is possible to
either store or compute which elements of the original input set \(\S\) match a
particular element of the sets \(\A,\B,\C\), allowing one to use the output of
the \twoSUM or \threeSUM algorithm run on the constructed instance to find the
output of the original \kLDT instance.

