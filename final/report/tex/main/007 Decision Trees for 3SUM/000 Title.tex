\chapter{Decision Trees for \threeSUM}
\label{tree:3sum}

In this chapter, we will talk about lower bounds for \threeSUM implied by a
reasoning on the decision tree model. The \threeSUM is of great interest
because of its applications to complexity theory. A myriad of problems have
been proved to be \threeSUM-hard, in other words a strong lower bound for the
\threeSUM problem would imply strong lower bounds for a multitude of other
problems.

For a long time we thought that it was not possible to solve the \threeSUM
problem in subquadratic time. This strong thought takes the form of a
conjecture named \emph{the \threeSUM conjecture}. In 2014, \citet*{gronlund:2014}
proved that this conjecture was false, pushing away the
boundaries and leaving us with new paths to explore.

On this topic, we will first illustrate an example of a \threeSUM-hard problem
and detail a linear-time mapping reduction from \threeSUM for this problem. In
the second section we will formally state the \threeSUM conjecture. The last
four sections of this chapter will be dedicated to results in the linear
decision tree model. The third one will explain results due to
\citet*{erickson:1999} commenting the lack of power of $k$-linear decision
trees when it comes to solving \ksum. Then we will cite results by
\citet*{ailon:2005} using $(k+1)$-linear decision trees.  In the fifth section
we will talk about recent results, due to \citet*{gronlund:2014}, that refute
the \threeSUM conjecture using $(2k-2)$-linear decision trees. Lastly we will
detail the method that allows to solve \kLDT using \threeSUM.
