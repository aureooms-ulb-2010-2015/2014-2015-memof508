\chapter{Decision Trees for \threeSUM}
\label{tree:3sum}

In this chapter, we discuss lower bounds for \threeSUM implied by a
reasoning on the decision tree model and also present more
general results for \kLDT. The \threeSUM problem is of great
interest because a myriad of problems
have been proved to be \threeSUM-hard. Hence, a strong lower bound for
the \threeSUM problem would imply strong lower bounds for a multitude of other
problems.

For a long time we thought that it was not possible to solve the \threeSUM
problem in subquadratic time. This took the form of a
conjecture named \emph{the \threeSUM conjecture}. In 2014, \citet*{gronlund:2014}
proved that this conjecture was false, pushing away the
boundaries and leaving us with new paths to explore.

On this topic, we first illustrate an example of a \threeSUM-hard problem
and detail a linear-time mapping reduction from \threeSUM for this problem. In
the second section we formally state the \threeSUM conjecture. The last
four sections of this chapter are dedicated to results in the linear
decision tree model. The third one explains results due to
\citet*{erickson:1999} commenting the lack of power of $k$-linear decision
trees when it comes to solving \ksum. Then we cite results by
\citet*{ailon:2005} using $s$-linear decision trees, where \(s > k\). In the fifth section
we discuss recent results, due to \citet*{gronlund:2014}, that refute
the \threeSUM conjecture using $(2k-2)$-linear decision trees. Lastly we
detail a method that allows to solve \kLDT using algorithms for \threeSUM.
