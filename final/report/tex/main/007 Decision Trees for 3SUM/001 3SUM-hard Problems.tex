\section{\threeSUM-hard problems}

We already mentioned that many problems are reducible from \threeSUM in such a
way that an efficient algorithm for those problems would provide efficient
algorithms for the \threeSUM problem. Hence, a lower bound for \threeSUM would
imply lower bounds for those other problems.

One of these problems is the \(3\)-points-on-line problem in the plane,

\begin{problem}[\(3\)-points-on-line problem]
Given a finite set \(\U\) of points in the plane, decide if \(\U\) contains
three collinear points.
\end{problem}

\begin{theorem}[\threeSUM-hardness of the \(3\)-points-on-line problem]
The three-set version of the \(3\)-points-on-line problem in the plane is \threeSUM-hard.
\end{theorem}

\begin{proof}
The three-set version of \threeSUM is reducible to a three-set version of
\(3\)-points-on-line by mapping all $a \in \A, b \in \B, c \in \C$ to the points
$a' = (a, a^3)$, $b' = (b, b^3)$, and $c' = (c, c^3)$. This mapping reduction
takes linear time in the input size. Moreover, there is a direct mapping from a
solution to the \(3\)-points-on-line instance that result of this reduction to
a solution of the original \threeSUM problem.

This works because if $a'$, $b'$, and $c'$ are collinear we have,

\begin{align*}
	a'_x ( b'_y - c'_y ) + b'_x ( c'_y - a'_y ) + c'_x ( a'_y - b'_y ) &= 0 \\
	a ( b^3 - c^3 ) + b ( c^3 - a^3 ) + c ( a^3 - b^3 ) &= 0 \\
\end{align*}

And when $ a + b + c = 0 \iff c = - ( a + b )$ we have,

\begin{align*}
	a b^3 + a ( a^3 + 3 a^2 b + 3 a b^2 + b^3 ) & \\
	- b ( a^3 + 3 a^2 b + 3 a b^2 + b^3 ) - a^3 b - a^4 - a^3 b + b^4 + a b^3 &= 0 \\
	\mathunderline{cyan}{a b^3} + \mathunderline{green}{a^4} +
	\mathunderline{blue}{3 a^3 b} + \mathunderline{red}{3 a^2 b^2} +
	\mathunderline{cyan}{a b^3} - \mathoverline{blue}{a^3 b} -
	\mathoverline{red}{3 a^2 b^2} & \\
	- \mathoverline{cyan}{3 a b^3} - \mathoverline{yellow}{b^4} -
	\mathoverline{blue}{a^3 b} - \mathoverline{green}{a^4} -
	\mathoverline{blue}{a^3 b} + \mathunderline{yellow}{b^4} +
	\mathunderline{cyan}{a b^3} &= 0 \\
	0 &= 0 \\
\end{align*}
\end{proof}

For a more exhaustive list, \citet*{king2004survey} and
\citet*{DBLP:journals/comgeo/GajentaanO12} propose a review of \threeSUM-hard
problems.
