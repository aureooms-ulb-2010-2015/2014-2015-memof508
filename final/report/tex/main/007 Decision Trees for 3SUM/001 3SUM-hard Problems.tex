\section{\threeSUM-hard problems}

We already mentioned that many problems are reducible from \threeSUM in such a
way that an efficient algorithm for those problems would provide efficient
algorithms for the \threeSUM problem. Hence, a lower bound for \threeSUM would
imply lower bounds for those other problems.

One of these problems is the \(3\)-points-on-line problem in the plane
\begin{problem}[\(3\)-points-on-line problem]
Given a finite set \(\U\) of points in \(\R^2\), decide if \(\U\) contains
three collinear points.
\end{problem}
\begin{theorem}[\threeSUM-hardness of the \(3\)-points-on-line problem]
The \(3\)-points-on-line problem in the plane is \threeSUM-hard.
\end{theorem}
\begin{proof}
A \threeSUM instance \(\S\) is reducible to
a \(3\)-points-on-line instance \(\S' = \enum{ (x,x^3) \st x \in \S}\). This reduction
takes linear time in the input size. Moreover, there is a direct mapping from a
solution to the \(3\)-points-on-line instance to
a solution of the original \threeSUM problem.

Indeed, if $a',b',c' \in \S'$ are collinear we have
\begin{align*}
a'_1 ( b'_2 - c'_2 ) + b'_1 ( c'_2 - a'_2 ) + c'_1 ( a'_2 - b'_2 ) &= 0\\
a ( b^3 - c^3 ) + b ( c^3 - a^3 ) + c ( a^3 - b^3 ) &= 0
\end{align*}
which if treated as a polynomial in \(c\) has roots \(\enum{-(a+b),a,b}\).
Roots \(a\) and \(b\) can be checked trivially.
When $ a + b + c = 0 \iff c = - ( a + b )$ we have,
\begin{align*}
a b^3 + a ( a^3 + 3 a^2 b + 3 a b^2 + b^3 )&\\
- b ( a^3 + 3 a^2 b + 3 a b^2 + b^3 ) - a^3 b - a^4 - a^3 b + b^4 + a b^3&\\
= \mathunderline{cyan}{a b^3} + \mathunderline{green}{a^4} +
\mathunderline{blue}{3 a^3 b} + \mathunderline{red}{3 a^2 b^2} +
\mathunderline{cyan}{a b^3} - \mathoverline{blue}{a^3 b} -
\mathoverline{red}{3 a^2 b^2}&\\
- \mathoverline{cyan}{3 a b^3} - \mathoverline{yellow}{b^4} -
\mathoverline{blue}{a^3 b} - \mathoverline{green}{a^4} -
\mathoverline{blue}{a^3 b} + \mathunderline{yellow}{b^4} +
\mathunderline{cyan}{a b^3}&= 0
\end{align*}
Since we will only check if distinct points are collinear we cannot have \(c = a\)
or \(c = b\). Hence, the only realisable root of our polynomial is \(c = -( a + b
) \).
\end{proof}

For a more exhaustive list, \citet*{king2004survey} and
\citet*{DBLP:journals/comgeo/GajentaanO12} propose a review of \threeSUM-hard
problems.
