\chapter{Decision Trees for $3$-SUM}

In this chapter, we will talk about lower bounds for $3$-SUM implied by
a reasoning on the decision tree model. The $3$-SUM problem having (almost) no
direct application to real world problems, the study of a lower bound for this
problem might seem childish. However, this would be true if not taking into
account the myriad of problem that have been proved to be $3$-SUM-hard. In
other words, a strong lower bound for the $3$-SUM problem would imply strong
lower bounds for a multitude of other problems.

For a long time we thought that it was not possible to solve the $3$-SUM
problem in subquadratic time. This strong thought takes the form of a
conjecture named \emph{the $3$-SUM conjecture}. In 2014, \citet*{gronlund:2014}
proved that this conjecture was false, pushing away the
boundaries and leaving us with new things to explore.

On this matter, we will first explain results due to
Erickson \cite{erickson:1999} commenting the lack of power of $k$-linear
decision trees when it comes to solving $k$-SUM. Secondly we will expose
results by Ailon and
Chazelle \cite{ailon:2005} using $(k+1)$-linear decision trees.
Lastly we will talk about recent results, due to Gronlund and
Pettie \cite{gronlund:2014}, that refutes the $3$-SUM conjecture using
$(2k-2)$-linear decision trees.
