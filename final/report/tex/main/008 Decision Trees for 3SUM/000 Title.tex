\chapter{Decision Trees for \threeSUM}
\label{tree:3sum}

In this chapter, we will talk about lower bounds for \threeSUM implied by
a reasoning on the decision tree model. The \threeSUM problem having (almost) no
direct application to real world problems, the study of a lower bound for this
problem might seem childish. This would be true if not taking into
account the myriad of problem that have been proved to be \threeSUM-hard. In
other words, a strong lower bound for the \threeSUM problem would imply strong
lower bounds for a multitude of other problems.

For a long time we thought that it was not possible to solve the \threeSUM
problem in subquadratic time. This strong thought takes the form of a
conjecture named \emph{the \threeSUM conjecture}. In 2014, \citet*{gronlund:2014}
proved that this conjecture was false, pushing away the
boundaries and leaving us with new paths to explore.

On this topic, we will first illustrate an example of a \threeSUM-hard problem
and its reduction. In the second section we will formally enunciate the
\threeSUM conjecture. The last 3 sections of this chapter will be dedicated to results
in the linear decision tree model. The third one will explain results due to
\citet*{erickson:1999} commenting the lack of power of $k$-linear
decision trees when it comes to solving \ksum. Fourthly we will cite
results by \citet*{ailon:2005} using $(k+1)$-linear decision trees.
Lastly we will talk about recent results, due to \citet*{gronlund:2014}, that
refutes the \threeSUM conjecture using $(2k-2)$-linear decision trees.
