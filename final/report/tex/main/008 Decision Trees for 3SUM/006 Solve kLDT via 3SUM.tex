\section{Solve \kLDT via \threeSUM}

In this section we will first show how to solve \kLDT for the case where \(k\)
is odd. After that, we can use a short argument to show that the same kind of
technique can be applied when \(k\) is even.

The standard three-set version of the quadratic \threeSUM algorithm for input
sets \(\A\), \(\B\) and \(\C\) completes after \BigO{\card{\C} ( \card{\A} +
\card{\B} ) } steps.

We will show how to solve any instance of a \kLDT problem with \(k \ge 3\) odd
in time \BigO{n^\frac{k+1}{2}}.

Given an input set \(\S\) of \(n\) real numbers, a set of \(k\) real
coefficient \( \enum{\alpha_1,\ldots,\alpha_k}\) and a real term \(\alpha_0\)
we construct the sets \(\A\), \(\B\) and \(\C\),

\begin{displaymath}
\A = \enum{\alpha_0 + \alpha_1 s_1 + \cdots + \alpha_{\frac{k-1}{2}} s_{\frac{k-1}{2}} \st s_i \in \S}
\end{displaymath}

\begin{displaymath}
\B = \enum{\alpha_{\frac{k+1}{2}} s_{\frac{k+1}{2}} + \cdots + \alpha_{k-1} s_{k-1} \st s_i \in \S}
\end{displaymath}

\begin{displaymath}
\C = \enum{\alpha_k s_k \st s_k \in \S}
\end{displaymath}

\dots
