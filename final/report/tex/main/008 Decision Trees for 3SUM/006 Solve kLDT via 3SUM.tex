\section{Solve \kLDT via \threeSUM}

In this section we will first show how to solve \kLDT for the case where \(k\)
is odd. After that, we can use a short argument to show that the same kind of
technique can be applied when \(k\) is even.

The standard three-set version of the quadratic \threeSUM algorithm for input
sets \(\A\), \(\B\) and \(\C\) completes after \BigO{\card{\C} ( \card{\A} +
\card{\B} ) } steps.

We will show how to solve any instance of a \kLDT problem with \(k \ge 3\) odd
in time \BigO{n^\frac{k+1}{2}}.

Given an input set \(\S\) of \(n\) real numbers, a set of \(k\) real
coefficient \( \enum{\alpha_1,\ldots,\alpha_k}\) and a real term \(\alpha_0\)
we construct the sets \(\A\), \(\B\) and \(\C\) as follows,

\begin{displaymath}
\A = \enum{\alpha_0 + \alpha_1 s_1 + \cdots + \alpha_{\frac{k-1}{2}} s_{\frac{k-1}{2}} \st s_i \in \S}
\end{displaymath}

\begin{displaymath}
\B = \enum{\alpha_{\frac{k+1}{2}} s_{\frac{k+1}{2}} + \cdots + \alpha_{k-1} s_{k-1} \st s_i \in \S}
\end{displaymath}

\begin{displaymath}
\C = \enum{\alpha_k s_k \st s_k \in \S}
\end{displaymath}

Each element of \(\A\) and \(\B\) has exactly \(\frac{k-1}{2}\) linear terms.
Elements of \(\A\) have and additional \(\alpha_0\) independent term. Looking
at all possible combinations of \(s_i\)'s with \(\alpha_j\)'s we know that
\(\card{\A} = \card{\B} = \BigO{n^{\frac{k-1}{2}}}\), whereas \(\card{\C} =
n\).

If we use the quadratic \threeSUM algorithm to solve the instance we
constructed, it will make \(k\)-linear queries since summing an element of
\(\A\) with an element of \(\B\) and comparing this sum to an element of
\(\C\) will involve \(2 \frac{k-1}{2} + 1\) linear terms. The running time of
the \threeSUM algorithm on this input is \(\BigO{n \cdot 2 \cdot n^{\frac{k-1}{2}}} =
\BigO{n^{\frac{k+1}{2}}}\).

Note that to use this algorithm we need to construct sets \(\A,\B,\C\).
Moreover we have to provide sorted structures for \(\A\) and \(\B\). However
the complexity of this preprocessing is only \BigO{n^{\frac{k-1}{2}} \log n}.

For \(k \ge 4\) even one constructs only two sets \(\A\) and \(\B\) of cardinality
\(\card{\A} = \card{\B} = \BigO{n^{\frac{k}{2}}}\), sorts them and searchs for
opposite values in the two sets. This procedure is \BigO{n^{\frac{k}{2}} \log
n} and is simply an algorithm solving instances of \twoSUM.

\citet*{gronlund:2014} explain that one can generalize their new decision tree for
\threeSUM from the one-set version to a three-set version. This version of the
decision tree is

\begin{displaymath}
\BigO{g(\card{\A}+\card{\B}) + g^{-1}\card{\C}(\card{\A}\card{\B})\log g}
\end{displaymath}

, hence their results improve the complexity of solving \kLDT for \(k\) odd to
\BigO{n^{\frac{k}{2}} \sqrt{\log n}} by choosing \(g = \sqrt{n \log n}\).
Also, we can note that this way of doing makes use of \(2k-2\)-linear queries.
Indeed \dots 
