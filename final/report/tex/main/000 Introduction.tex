\setcounter{chapter}{-1}
\chapter{Introduction}


Dear reader,


This work focuses on studying the complexity of problems and solving problems
in the decision tree model. In the model we will use, solving a problem will
amount to retrieve information. We will be allowed to retrieve information
through an omniscient oracle by means of questions that, in the general case,
can be answered ``yes'' or ``no''. The kind of question we will ask ourselves
about a problem is: ``For an instance of size \(n\), in the worst case, what is
the minimum number of questions we would have to ask to the oracle to solve our
problem''. We will have no interest in the storage of information we retrieve.
In other words, in our model, the only operation having a non-zero cost is
asking the oracle to answer one of our ``yes/no'' answerable questions.


In the first two chapters, we give a rehearsal on well-studied problems:
sorting and merging. The goal of this rehearsal will be to clearly define these
two basic problems as well as the model we will be working with. For example,
we will demonstrate non-uniformity of the decision tree model. This rehearsal
will also give us the occasion to introduce tools that will be needed all along
this document. Two examples of such tools are Stirling's approximation and the
mergesort algorithm.  Since all the problems we will approach are in a way or
another related to sorting, these two first chapters are really the starting
point of our journey.


The third chapter will be dedicated to formalization of commonly used objects
in order theory. In this chapter we will evolve crescendo from very basic
concepts to useful tools. We will first familiarize ourselves with order
relations and partially ordered sets. We will introduce Hasse diagrams that
will be used to represent partially ordered sets throughout this document.
This chapter will also contain information about structural particularities
that can be found in certain families of posets. The conclusion of this
chapter will consist in the introduction of graph entropy and consequently
poset entropy, one of the most useful tool we will be using to explain the
results exposed in chapter four.


Sorting under Partial Information

History

Linial's algorithm: merging under partial information

Extending the model: k-linear decision trees

X+Y, Set sum problems

Realtion between sorting and those problems

Detail Sorting X + Y
- importance of exploiting known information
- limitations of the poset approach
- generic algorithm of fredman
- counting linear extensions

Detail 3SUM
- recent results refuting a conjecture

Meiser's algorithm
- solving kSUM family problems by modeling them as point location problems
- non-uniformity of the model bis
- trick to use only \(o(n)\)-linear queries
