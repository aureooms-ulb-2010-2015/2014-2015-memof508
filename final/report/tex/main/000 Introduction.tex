%\setcounter{chapter}{-1}
\chapter*{Introduction}
\addcontentsline{toc}{chapter}{Introduction}%

This work focuses on studying the complexity of problems and solving problems
in the decision tree model. In the model we use, solving a problem
amounts to retrieve information. We are allowed to retrieve information
through an oracle by means of questions that, in the general case,
can be answered ``yes'' or ``no''. The kind of questions we ask ourselves
about a problem is: ``For an instance of size \(n\), in the worst case, what is
the minimum number of questions we need to ask the oracle to solve our
problem?''. We have no interest in the storage of information we retrieve.
In other words, in our model, the only operation having a non-zero cost is
asking the oracle to answer one of our ``yes/no'' questions.

In the first two chapters, we give an overview on two well-studied problems:
sorting and merging. The goal of this overview is to clearly define these two
basic problems as well as the model we are working with. For example, we point
out the nonuniformity of the decision tree model. This overview also gives us
the opportunity to introduce tools that are needed all along this document. Two
examples of such tools are Stirling's approximation of the factorial
\cite{feller1967direct} and the \mergesort algorithm
\cite{goldstine:1948,leiserson:2001}. Since all the problems we approach are in
a way or another related to Sorting, these first two chapters are the starting
point of our work.

The third chapter is dedicated to formalization of commonly used objects
in order theory. We first familiarize ourselves with order
relations and partially ordered sets. We introduce Hasse diagrams that
are used to represent partially ordered sets throughout this document.
This chapter also contains information about the structure of
certain families of posets. The conclusion of this
chapter consists in the definition of graph entropy \cite{korner1973coding}
and poset entropy, one of the most useful tools we use to explain the
results exposed in Chapter \(4\).

In the fourth chapter, we tackle the problem of Sorting under Partial
Information. Sorting under Partial Information, abbreviated \concept{SUPI}, is
a more general problem than sorting. When we want to sort an array of integers
with the \quicksort algorithm for example, we assume the array is shuffled and
that no prior information can be extracted from the input without asking the
oracle. In SUPI we do have some initial information on the input, modeled as a
poset, and we explain ways
to find the linear extension of a poset using a smaller number of queries than
what would be required by classical sorting algorithms. We first define
the problem and explain its origins. We then explain techniques that can
be used to solve this problem \cite{kahn:1995,cardinal:2013}. The
last two sections of this chapter concern
algorithms that solve the problem of Merging under Partial Information or
\concept{MUPI}, a special case of Sorting under Partial Information. We first
give a detailed explanation of an algorithm due to \citet*{linial:1984} and
then give an original, more practical implementation of this algorithm. This
new implementation solves an MUPI instance with less than \(1.44 \log e(\P)\)
comparisons and has a time complexity of \BigO{n^2 + n \log e(\P)}. The
\(n^2\) term of the time complexity measure can be moved to a preprocessing
phase if \(\P\) is known in advance.

The fifth chapter introduces a variety of problems that are all related to
each other. We introduce the classical set sum problems that are
\threeSUM, \kSUM, and subset sum. Other problems include a generalization of
\kSUM known as linear degeneracy testing, abbreviated \kLDT, and sorting \XY.
We explain how one can establish connections between those problems and the problem
of sorting. We also need to upgrade our decision tree model and for this
purpose we introduce the notion of \(k\)-linear decision trees. This
extended model is used in the last three chapters.

The sixth chapter focuses on the problem of sorting \XY. We make
multiple attempts to solve this problem efficiently and evaluate its complexity
using two different sets of tools. We show how exploiting known information
can be significant and the limitations of the partially ordered set tools that
were used for solving the Sorting under Partial Information problem. We
detail one of the first algorithms designed to solve \concept{SUPI}
\cite{fredman:1976} and
show that using this algorithm, one can solve the problem of sorting \XY
using less comparisons than with the poset theory approach.

\threeSUM is the subject of the seventh chapter. We introduce the
subject by demonstrating the interest of studying the \threeSUM problem. Then,
we state a long-lived but freshly refuted conjecture concerning the
complexity of \threeSUM, and explain the implications. The third part
consists of three sections. The first two contain previous results while
the third one exposes recent results that refute the above-mentioned
conjecture \cite{gronlund:2014}. Finally, we explain how these results can be applied to solve
the more generic \kLDT problem.

The last chapter concerns the application of a point location algorithm to
the family of \kSUM problems \cite{meiser:1993,burgisser:1997}. We show how one can model those subset sum
problems as point location problems. We then detail the algorithm that
can solve these newly modeled problems. We have the opportunity to insist again
on the nonuniformity of the decision tree model. We end our dissertation
with an original trick that limits the size of queries to the oracle when
using our point location algorithm.
