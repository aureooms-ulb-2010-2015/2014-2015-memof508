%\setcounter{chapter}{-1}
\chapter*{Introduction}
\addcontentsline{toc}{chapter}{Introduction}%

This work focuses on studying the complexity of problems and solving problems
in the decision tree model. In the model we will use, solving a problem will
amount to retrieve information. We will be allowed to retrieve information
through an oracle by means of questions that, in the general case,
can be answered ``yes'' or ``no''. The kind of question we will ask ourselves
about a problem is: ``For an instance of size \(n\), in the worst case, what is
the minimum number of questions we would have to ask to the oracle to solve our
problem?''. We will have no interest in the storage of information we retrieve.
In other words, in our model, the only operation having a non-zero cost is
asking the oracle to answer one of our ``yes/no'' answerable questions.

In the first two chapters, we give a reminder on well-studied problems:
sorting and merging. The goal of this reminder will be to clearly define these
two basic problems as well as the model we will be working with. For example,
we will point out the nonuniformity of the decision tree model. This reminder
will also give us the opportunity to introduce tools that will be needed all along
this document. Two examples of such tools are Stirling's approximation and the
Mergesort algorithm.  Since all the problems we will approach are in a way or
another related to sorting, these two first chapters are really the starting
point of our journey.

The third chapter will be dedicated to formalization of commonly used objects
in order theory. In this chapter we will evolve crescendo from very basic
concepts to useful tools. We will first familiarize ourselves with order
relations and partially ordered sets. We will introduce Hasse diagrams that
will be used to represent partially ordered sets throughout this document.
This chapter will also contain information about structural particularities
that can be found in certain families of posets. The conclusion of this
chapter will consist in the introduction of graph entropy and consequently
poset entropy, one of the most useful tool we will be using to explain the
results exposed in chapter four.

In the fourth chapter, we will tackle the problem of sorting under partial
information. Sorting under Partial Information, abbreviated \concept{SUPI}, is
a more generic problem than sorting. When we want to sort an array of integers
with the Quicksort algorithm for example, we assume the array is shuffled and
that no prior information can be extracted from the input without bothering the
oracle. Here, we will explain ways to exploit some already existing information
to find the linear extension of a poset using a smaller number of queries than
what would be required by classical sorting algorithms. We will first define
the problem and explain its origins. We will then explain techniques that can
be used to solve this problem. The last part of this chapter will concern an
algorithm that solves the problem of Merging under Partial Information or
\concept{MUPI}, a special case of Sorting under Partial Information.

The fifth chapter will introduce a variety of problems that are all related to
each other. We will introduce the classical set sum problems that are
\threeSUM, \kSUM and subset sum. Other problems include a generalization of
\kSUM known as linear degeneracy testing, abbreviated \kLDT, and sorting \XY.
We will explain how one can make links between those problems and the problem
of sorting. We will also need to upgrade our decision tree model and for this
purpose we will introduce the notion of \(k\)-linear decision trees. This
extended model will be of use for the three last chapters.

The sixth chapter will focus on the problem of sorting \XY. We will make
multiple attempts to solve this problem efficiently and evaluate its complexity
using two different set of tools. We will see how exploiting known information
can be significant and the limitations of the partially order set tools that
were used for solving the Sorting under Partial Information problem. We will
detail one of the very first algorithms designed to solve \concept{SUPI} and
show that using this algorithm, one can solve the problem of sorting \XY
optimally in terms of comparisons.

\threeSUM will be the subject of the seventh chapter. We will introduce the
subject by demonstrating the interest of studying the \threeSUM problem. Then,
we will state a long lived but freshly refuted conjecture concerning the
complexity of \threeSUM and explain its implications. The third part will
consist of three sections. The first two will contain previous results while
the third one will expose recent results that refute the above-referenced
conjecture. Finally, we will explain how these results can be applied to solve
the more generic \kLDT problem.

The last chapter will concern the application of a point location algorithm to
\kSUM family problems. We will show how one can model those subset sum
problems as point location problems. We will then detail the algorithm that
can solve these newly modeled problems. We will have the opportunity to insist again
on the nonuniformity of the decision tree model. We will end our dissertation
with a little trick that limits the size of queries send to the oracle when
using our point location algorithm.
