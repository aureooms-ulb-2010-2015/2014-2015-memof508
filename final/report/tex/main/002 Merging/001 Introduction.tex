\label{tree:merging:intro}

If we ignore the merge sort algorithm which was presented in
\ref{tree:sorting:alg/table} there are not so many example of practical merging
procedures. Indeed, this problem seems impopular and outdated. Even
\emph{Wikipedia} states that merge algorithms are \emph{a family of algorithms
that run sequentially over multiple sorted lists, typically producing more
sorted lists as output}, which are not the only algorithms that we will
consider. If we want for example to build a sorted output sequence from two
sorted input sequences, a common, practical and simple way to do it is to read
both input sequences sequentially, as if they were priority queues, iteratively
removing the smallest element out of the two input sequences heads and adding
it to the end of the output sequence. This approach has \BigO{n} complexity for
both comparisons and other computation operations. In this chapter, we will
analyze the number of comparison operations really needed to merge two totally
ordered sets. Here, we will define the ITLB as some formula expressing a lower
bound on the number of comparisons required to merge two totally ordered sets
of cardinality $m$ and $n$. We will then find that there exists an algorithm
whose time complexity is \BigO{ITLB}.

