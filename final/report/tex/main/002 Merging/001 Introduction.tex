\label{tree:merging:intro}

If we ignore the \mergesort algorithm which was presented in
the previous chapter there are not so many examples of practical merging
procedures. Indeed, this problem seems unpopular and outdated. For example,
\emph{Wikipedia} states that merging algorithms are \emph{a family of algorithms
that run sequentially over multiple sorted lists, typically producing more
sorted lists as output}, which are not the only algorithms that we
consider.

We explain our motivation with the following observation.
If we want to build a sorted output sequence from two sorted input sequences, a
common, practical and simple way to do it is to read both input sequences
sequentially, as if they were priority queues, iteratively removing the
smallest element out of the two input sequences heads and adding it to the end
of the output sequence. This algorithm is the merge procedure of the \mergesort
algorithm and is called \tapemerge. \tapemerge has \BigO{n} complexity for
both comparisons and other computation operations. However, the query
complexity of \tapemerge does not match the ITLB of the merging problem for all
inputs. In this chapter, we explain how we can beat the \tapemerge algorithm
with cleverer techniques and how to apply these results to the problem of
merging \(k\) sorted sequences.

In the first section, we give a formal statement of the merging problem, \ie
the problem of merging \(k\) sorted sequences of different sizes. In the second
section, we analyze the ITLB for the problem of merging two sorted sequences of
different sizes and give a description of the Hwang-Lin algorithm
\cite{DBLP:journals/siamcomp/HwangL72} that solves this problem using
\BigO{ITLB} queries. In the last section, we analyze the ITLB of the merging
problem and give a \BigO{ITLB} algorithm solving this problem. This last
algorithm is built on top of the Hwang-Lin algorithm.
