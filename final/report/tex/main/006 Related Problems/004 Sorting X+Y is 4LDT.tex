\section{Sorting \XY is a \fourLDT problem}

Similarly to how the sorting problem is a linear degeneracy testing problem,
sorting $X+Y$ is a \fourLDT problem. Let us define the \fourLDT problem,

\begin{problem}
Given a set $S \subset \R$, $\card{S} = n$, and real coefficients $\alpha_0,
\alpha_1, \alpha_2, \alpha_3, \alpha_4$ find all $(a,b,c,d) \in S^4$ such that
$\alpha_0 + \alpha_1 a + \alpha_2 b + \alpha_3 c + \alpha_4 d = 0$.
\end{problem}

When sorting $X+Y$ we need to determine whether $a + b < c + d$ for all
$(a,b,c,d) \in (X \times Y)^2$, this sums up to determining the sign of $(a+b) -
(c+d)$, i.e. asking questions of the kind $a + b - c - d = 0$. We can thus
reformulate sorting $X+Y$ the following way,

\begin{problem}
Given a set $S \subset \R$, $\card{S} = n$, find all $(a,b,c,d) \in S^4$ such
that $a + b - c - d = 0$.
\end{problem}

Again, we can construct an adversary argument to prove that deciding whether
there exist \(a,b,c,d\) such that \(a+b-c-d=0\) requires signs of all
\(a+b-c-d\) to be found.

All sorting $X+Y$ problem instances are thus instances of the \fourLDT problem
with $\alpha_0 = 0, \alpha_1 = \alpha_2 = 1, \alpha_3 = \alpha_4 = -1$ and
$S = X \cup Y$.

