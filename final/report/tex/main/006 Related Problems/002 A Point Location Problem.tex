\section{A Point Location Problem}

We now describe a problem which, we will see later, can be used to model
various kind of set sum problems, including subset sum, \kSUM and \kLDT.

The problem is to locate a point $x$ in the $n$-dimensional space relatively
to an arrangement of hyperplanes $\A(\H)$. This arrangement divides the space
in cells, cones, and faces and the goal will be to determine which
cell, cone, or face contains point $x$. This amounts to determine for each
hyperplane $H_i$ whether point $x$ lies above, on or below $H_i$.

Formally, we define the point location in an arrangement of hyperplanes as
follows,

\begin{problem}
Given a point $x \in \R^n$ and an arrangement of hyperplanes $\H = \enum{ H_i
\st 1 \leq i \leq m }$ determine the position vector of $x$, $\pv(x) \in
\signset^n$ where $\pv_i(x) = \sigma, \sigma \in \signset$ iff $x \in
H_i^{\sigma}$. We define $H_i^{0}$, $H_i^{-}$ and $H_i^{+}$ to be
respectively $H_i$ itself, the half-space under $H_i$ and the half-space above
$H_i$.
\end{problem}

There exists several methods to solve this problem. However, we will focus on
Meiser's Algorithm~\cite{meiser:1993}, since it is the first algorithm that
solves this point location problem in time polynomial in both $n$ and $\log
m$, for fixed $\H$ and $n$. This particular algorithm will give us the
occasion to prove that the decision tree complexity of subset sum is
polynomial.
