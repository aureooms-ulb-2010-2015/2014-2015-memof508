\section{A Point Location Problem}


\begin{table}
	\begin{center}
	\caption{Algorithms for the point location in an arrangement of hyperplanes
problem}
	\label{tree:sortsumldt:other:pointlocation/algorithms}
	\begin{tabular}{|c|c|c|c|}

	\hline
	Name & Preprocessing & Time-Complexity & Query-Complexity\\\hline\hline
	Meiser's Algorithm & yes & exponential & \BigO{n^3}\\\hline
	Meiser's Algorithm & yes & exponential & \BigO{n^3}\\\hline
	Meiser's Algorithm & yes & exponential & \BigO{n^3}\\\hline
	Meiser's Algorithm & yes & exponential & \BigO{n^3}\\\hline
	\end{tabular}
	\end{center}
\end{table}


We now describe a problem which, we will see later, can be used to model
various kind of set sum problems, including linear degeneracy testing problems.

We define the point location in an arrangement of hyperplanes as follows,

\begin{problem}
Given a point $x \in \R^n$ and an arrangement of hyperplanes $\H = \enum{ H_i
\st 1 \leq i \leq m }$ determine the position vector of $x$, $\pv(x) \in
\signset^n$ where $\pv_i(x) = \sigma, \sigma \in \signset$ iff $x \in
H_i^{\sigma}$. We define $H_i^{0}$, $H_i^{-}$ and $H_i^{+}$ to be
respectively $H_i$ itself, the half-space under $H_i$ and the half-space above
$H_i$.
\end{problem}

There exist several methods to solve this problem. We classify them
in~\ref{tree:sortsumldt:other:pointlocation/algorithms}.
