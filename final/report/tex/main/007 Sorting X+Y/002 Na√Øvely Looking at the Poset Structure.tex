\section{Naïvely Looking at the Poset Structure}
\label{tree:xy:grid}

In the \XY problem, we are asked to find a linear extension of a known partial
order having a specific structure. The \XY problem is thus a special case of
Sorting under Partial Information. In \ref{tree:supi}, we have seen that there
are algorithms solving the SUPI problem in \BigO{\log e(P)} comparisons. We
wonder if it is possible to prove that, if we are only allowed to exploit the
grid structure of a \XY poset, then \XY cannot be solved in less than \BigO{n^2
\log n} comparisons.

We show that the number of linear extension of the \( n \times n \) lattice is,

\begin{displaymath}
e(P) = \frac{(n^2)!}{\prod_{k=1}^{2n-1} k^{n - \abs{n-k}}}
\end{displaymath}

note that one can use the superfactorial notation
\( \sf(n) = n!\,(n-1)!\,\ldots 3!\,2!\,1! \)
to make the formula slimmer,

\begin{displaymath}
e(P) = \frac{(n^2)! \, \sf(n-1)^2}{\sf(2n-1)}
\end{displaymath}

To do so, we introduce the notion of Ferrers Diagrams and Young Tableaux. A
\dots

The proof is due to \citet*{greene:1979},

\begin{proof}

\end{proof}

Hence for a \( n \times n \) Young tableau, i.e. a Young tableau of shape
\( \lambda = \enum{ \lambda_1 , \ldots , \lambda_n } \)
with
\( \lambda_i = n, \Forall i = 1 \ldots n \)
, we obtain
\( d_{\lambda} = e(P) \).

We still need to show that counting tableaux is the same as counting linear
extensions. We etiquette all \( n^2 \) nodes of \(P\), the Hasse Diagram of
\XY, with the numbers \(1\) to \( n^2 \) in any order. Suppose that we are
given a unique linear extension of \(P\) that can be represented as a tuple
\( \omega = ( j_1 , \ldots , j_{n^2}) \)
with all \( j_i \)'s distinct and between \(1\) and \( n^2 \) using our
etiquettes. If we are given \( n^2 \) distinct elements \( x_1 \) to
\( x_{n^2} \) then according to our proof, there are \( d_{\lambda} \) ways to
fill our tableau with those elements and to each filled tableau corresponds a
unique linear extension designated by \(\omega\).

\dots

Note that we also find those numbers on the diagonal of the matrix of
multidimensional Catalan numbers \cite{OEIS:A060854}, i.e. the number of
shortest paths in the \(n\)-dimensional unit grid from \( (0,\ldots,0) \) to
\( (n,\ldots,n) \) such that if \( (x_1,\ldots,x_n) \) is on a path then
\( x_1 \le x_2 \le \ldots \le x_n\).

We show that \( \log e(P) = \BigO{n^2 \log n} \) \dots

\begin{displaymath}
\lim_{n \to \infty} \ln e(P) \ge n^2 \ln n - 2n^2 + \ln n + \frac{1}{2} \ln \pi + \frac{1}{2}
\end{displaymath}


Thus, solving \XY in \SmallO{n^2 \log n} comparisons is not possible
if we only consider the information contained in the Hasse diagram structure.
\ref{tree:xy:counting} shows that there is indeed more information available.
