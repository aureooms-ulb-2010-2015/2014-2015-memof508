\section{Basic Operations, Subposets, and Intervals}
\label{tree:poset:sub}


Since a poset is the combination $(\S, \le_{\S})$ of a set $\S$ and an order
$\le_{\S}$, we might be interested in applying standard set operations to posets.

The binary operations Union ($\cup$), Intersection ($\cap$) and Complement
($\setminus$) for posets $\P = (\S, \le_{\S})$ and $\Q = (\T, \le_{\T})$ only make sense
when $\le_{\S} = \le_{\T}$. For example, when we run the algorithm \mergesort at each
recursion step we merge (read ``make the union of'') two posets having the same
total order. We thus only define those operations for posets having the same
order relation, and the result of any of those three binary operations on $\P$
and $\Q$ is a new poset $(\U, \le_{\U})$, where $\U = \S~\text{op}~\T$,
\(\text{op} \in \enum{\cup,\cap,\setminus}\) and $\le_{\U} =
\le_{\S} = \le_{\T}$.


In the same way we handled the binary operations we might look at how we can
define what a subposet is. For what follows, $\P = (\S, \le_{\S})$ is a poset
and $\Q = (\T, \le_{\T})$ a subposet of $\P$.

Unlike the binary operations, we might consider the case $\le_{\T} \neq
\le_{\S}$. Two main types of subposets emerge: subposets for which $\le_{\T} =
\le_{\S}$ and others. The former are called induced subposets and the latter
weak subposets.
\begin{definition}[Induced subposet]
By an induced subposet of $\P$, we mean a subset $\Q$ of $\P$ and a partial
ordering of $\Q$ such that for $s, t \in \Q$ we have $s \leq t$ in $\Q$ if and
only if $s \leq t$ in $\P$. We then say the subset $\Q$ of $\P$ has the induced
order.
\end{definition}

This definition is similar to the definition of an induced subgraph. If we
represents a poset $\P$ as a directed graph $G$ (see
\Cref{tree:poset:graph}), each edge represents an element of an order relation
and when we remove a vertex from one of those graphs we also have to remove
incident edges since one of their endpoints is missing. Considering that $\P$ is
finite, there are $2^{\card{V(G)}}$ induced subgraphs of $G$, thus the poset $\P$ has
exactly $2^{\card{\P}}$ induced subposets. Note that when using the expression
``a subposet of $\P$'', we always mean ``an induced subposet of $\P$''.

The set of weak subposets is a superset of the set of induced subposets where
arbitrary edge deletion in the graph associated with a poset is allowed.
All induced subposets are also weak subposets.
\begin{definition}[Weak subposet]
A weak subposet of $\P$ is thus a subset $\Q$ of the elements of $\P$ and a
partial ordering of $\Q$ such that if $s \leq t$ in $\Q$, then $s \leq t$ in $\P$.
If $\Q$ is a weak subposet of $\P$ with $\P = \Q$ as sets, then we call $\P$ a
refinement of $\Q$ (only the order relations differ).
\end{definition}

We give the remaining definitions for the sake of completeness.
\begin{definition}[Closed interval]
A special type of subposet of $\P$ is the (closed) interval $[s, t] = \left\{{u
\in \P \st s \leq u \leq t}\right\}$, defined whenever $s \leq t$. The interval
$[s, s]$ consists of the single point $s$.
By this definition the empty set is not regarded as a closed interval since
partial order relations are reflexive.
\end{definition}
\begin{definition}[Open interval]
We give a similar definition to the open interval.
The open interval $(s, t) = \left\{{u \in \P \st s < u <
t}\right\}$, so $(s, s) = \emptyset$.
\end{definition}
\begin{definition}[Locally finite poset]
If every interval of $\P$ is finite, then $\P$ is called a locally finite poset,
\eg $\P = (\N, \le)$ is locally finite while $\Q = (\R, \le)$ is
not.
\end{definition}
\begin{definition}[Convex subposet]
We also define a subposet $\Q$ of $\P$ to be convex if $t \in \Q$ whenever $s < t
< u$ in $\P$ and $s, u \in \Q$. Thus, an interval is convex. The difference with a
closed interval is that the number of minimal/maximal elements can be greater than 2.
\end{definition}

%TODO figure of convex, interval
