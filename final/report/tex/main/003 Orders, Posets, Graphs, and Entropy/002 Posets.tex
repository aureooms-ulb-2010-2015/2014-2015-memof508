\section{Posets}

The word poset stands for \emph{Partially Ordered Set}. Starting from
what we already know, the only new term is ``partial''. ``Partial'' can
both reflect our ignorance about a more
``complete\footnote{Do not mix completeness of partial orders with the informal
``completeness'' we are referring to, ``complete'' and ``incomplete''
are words that help to express what we mean by partial.}''
order on our data set
and the ``incomplete'' nature of the ordered set we consider. By
``incomplete'' we mean that there could exist some $x, y \in \S$ for which
neither $(x, y) \in G$ nor $(y, x) \in G$. We say that $(x,y)$ is an
\emph{incomparable pair} of the partially ordered set \(\S\). When however a partial
order is ``complete'', we say that it is a \emph{total order}.
\begin{definition}[Partially ordered set]
A partially ordered set $\P$ is a set (which by abuse of notation we also call
$\P$), together with a binary relation denoted $\leq$ (or $\leq_{\P}$ when there is
a possibility of confusion), satisfying the following three axioms:
\begin{enumerate}
\item For all $t \in \P, t \leq t$ (reflexivity);
\item If $s \leq t$ and $t \leq s$, then $s = t$ (antisymmetry);
\item If $s \leq t$ and $t \leq u$, then $s \leq u$ (transitivity).
\end{enumerate}

We say that two elements $s$ and $t$ of $\P$ are comparable if $s \leq t$ or $t
\leq s$; otherwise, $s$ and $t$ are incomparable, denoted $s \parallel t$.

We use the obvious notation $t \geq s$ to mean $s \leq t$, $s < t$ to mean $s
\leq t$ and $s \neq t$, and $t > s$ to mean $s < t$.
\end{definition}

Taking other properties aside, the most interesting feature of a partial order
(with respect to our point of view) is the transitivity axiom it must satisfy.
Let us explain why it is interesting. The goal of sorting and merging is to,
starting from a partially ordered set, find the underlying total order. To find
this total order, we are allowed to query an oracle that answers to ``yes/no''
questions of the type ``is $x~R\text{-related to}~y$?'' in the unknown total
order. This has to be done efficiently. By efficiently we mean that we do not
want to ask more questions than necessary.

Before continuing further, note also that since we are trying to find a total
order, at least one of the questions $(x, y) \ask{\in} G$ and $(y, x)
\ask{\in} G$ has a positive answer. Hence, we can already discard
half of all the questions we could ask.

Once we have restricted the number of useful questions to approximately
half of all possible questions, there are still \BigO{n^2} questions we
might want to ask. That is where transitivity comes into play. Without it we
would have had to ask all the $\frac{n (n-1)}{2}$ questions to unveil the total
order.

For example, due to the transitivity axiom, every time we ask the two questions
$(x, y) \ask{\in} G$ and $(y, z) \ask{\in} G$ and both give the
same answer it is not necessary anymore to ask the third question $(x, z)
\ask{\in} G$.

This is what we do when we sort a list of elements
efficiently with \mergesort or any other \BigO{n \log n} comparison sort
algorithm. We exploit the transitive nature of partial orders.
