\section{Linear Extensions and Entropy}

A linear extension of a partial order $\leq$ is a total order $\leq^*$
compatible with $\leq$. \emph{Linear} means that the order $\leq^*$ must be
total while \emph{extension} means that the original order $\leq$ is
``preserved'' \ie when \(x \leq^* y\) for all pair \((x,y)\) such that \(x
\leq y\).

In our decision tree model, the number of linear extensions $e(\P)$ of a poset
$\P$ is strongly related to the information contained in this poset. Since the
total information is $\log n!$, the information we miss is $\log e(\P)$, then
the information we have is $\log n! - \log e(\P)$.

Computing $e(\P)$ in the general case is \#P-complete
(see \citet*{brightwell1991counting}). In the next chapter we talk about a
problem called Sorting under Partial Information (\SUPI). This problem can be solved using \BigO{\ITLB}
queries by an algorithm that
iteratively chooses a `good' query. We will see later that a `good' query \(a \ask{\le} b\) can be
found by computing the ratio of \(e(\P(a \le b))\) to \(e(\P)\) for all \((a,b)\) pairs, where
\(\P(a \le b)\) denotes the poset obtained
from \(\P\) after adding the constraint that \(a \le b\).
Since it is unlikely that
a polynomial-time algorithm for computing \(e(\P)\) exists, it is unlikely that the above algorithm
can be implemented such that it runs in time polynomial in the input size.

Fortunately, \citet*{kahn:1995} give a good
approximation of $e(\P)$ using the definition of entropy of a poset we
introduced earlier,

\begin{theorem}[\citet*{kahn:1995}]
\label{eq:khankim1}
For any poset \(\P\) of order \(n\),
\begin{displaymath}
\log e(\P) \le n H(\widetilde{\P}) \le \min\enum{\log e(\P) + \log e \cdot n, c_1
\log e(\P)},
\end{displaymath}
where \(c_1 = (1 + 7 \log e) \approx 11.1\).
\end{theorem}

It is a good approximation because we have the guarantee that
\(n H(\widetilde{\P}) = \BigTheta{\log e(\P)}\) and because we can compute
\(H(\widetilde{\P})\) in polynomial time by solving the convex minimization
problem given by \ref{eq:entropy:graph}.

\citet*{cardinal:2013} improve the results of \ref{eq:khankim1} by showing that
one can take \(c_1 = 2\), which is best possible.

\begin{theorem}[\citet*{cardinal:2013}]
For any poset \(\P\) of order \(n\),
\begin{displaymath}
n H(\widetilde{\P}) \le 2 \log e(\P).
\end{displaymath}
\end{theorem}

In the next chapter we will see what kind of algorithms can be built upon
these results.
