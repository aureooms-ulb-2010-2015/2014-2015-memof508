\section{Special Structures}

We give names to special structures and properties a poset might carry.

\subsection{Chains and Antichains}

We describe two special structures that can be found in
posets. Those two structures are chains and antichains.
They convey the intuitive idea of respectively \emph{a sorted sequence of
elements} and \emph{a bag of incomparable elements}.

The words chain and antichain are not the only one used.
Indeed, they have many synonyms; a chain can also be called
totally ordered set or linearly ordered set and an
antichain is also called Sperner family or
clutter.


Formally, a chain is a poset in which any two elements are
comparable. Thus in \sref{ex:poset:def:a}, the poset $\bm{n}$ and all of its
subposets are chains. A subset $\C$ of a poset $\P$ is called a chain if $\C$ is a
chain when regarded as a subposet of $\P$.

Similarly, we define an antichain as a subset $\A$ of a poset $\P$ such
that every two distinct elements of $\A$ are incomparable.

\subsection{Maximal and Saturated Chains}

We say that a chain \(\C\) of \(\P\) is maximal if it is not contained in a larger
chain of $\P$. For example in \sref{ex:poset:def:a}, $[1, n] = \bm{n}$ is the
only chain of $\bm{n}$ that is maximal.

We call a chain $\C$ of $\P$ saturated (or unrefinable) if there does not exist
$y \in \P \setminus \C$ that we could insert in chain $\C$. If $x < y < z$ for
some $x, z \in \C$ and $\C \cup \left\{{y}\right\}$ is a chain then we can insert
$y$ in $\C$ and $\C$ is not saturated. In \sref{ex:poset:def:a}, the poset
$\bm{n}$ and all of its intervals are saturated chains.

Maximal chains are thus saturated, however the converse is not true. In a
locally finite poset, a chain $x_0 < x_1 < \cdots < x_n$ is saturated if and
only if $x_{i-1}$ is covered by $x_i$ for $1 \le i \le n$.


\subsection{Lengths and Ranks}

Chains can be thought of as ordered sequences, and this notion of sequence makes
us want to introduce the concept of length of a chain. We define the length
$\ell(\C)$ of a finite chain by $\ell(\C) = \card{\C} - 1$. Using this definition, we
express the length (or rank) of a finite poset $\P$ by the following formula
\begin{displaymath}
\ell(\P) \bydef \max\left\{{\ell(\C) \st \C ~\text{is a chain of}~ \P}\right\}.
\end{displaymath}

The length of an interval $[x, y]$ is denoted $\ell(s, t)$. In the case where
every maximal chain of $\P$ has the same length $n$ we say that $\P$ is graded of
rank $n$. In this case there is a unique rank function $\rho \colon \P \to \left\{{0,
1, \ldots , n}\right\}$ such that $\rho(x) = 0$ if $x$ is a minimal element of
$\P$ ($x \le y$ or $y \nleq x$ for all $y \in \P$), and $\rho(t) = \rho(s) + 1$
if $t \gtrdot s$ in $\P$. If $s \le t$ then we also write $\rho(s, t) = \rho(t)
- \rho(s) = \ell(s, t)$. If $\rho(s) = i$, then we say that $s$ has rank $i$.
This means that we can associate a rank to every element in the poset $\P$.


\subsection{Order Ideals and Dual Order Ideals}

Here we look at structures that can be intuitively understood as generated by
selecting some subset $\I'$ of elements of a poset $\P$ and then extending this
selection by iteratively including all elements $y \in \P$ that
are either above an element of $\I'$ or below an element of $\I'$. These
structures are respectively called order ideals and dual order ideals.

Again, many other words exist to refer to those structures. Synonyms for
order ideal include semi-ideal, down-set and
decreasing subset. Synonyms for dual order ideal include
up-set, increasing subset and filter.

Formally we have the following two definitions
\begin{definition}[Order ideal]
An order ideal of $\P$ is a subset $\I$ of $\P$ such that if $x \in
\I$ and $y \le x$, then $y \in \I$.
\end{definition}
\begin{definition}[Dual order ideal]
A dual order ideal is a subset $\I$ of $\P$ such that if $x \in \I$
and $y \ge x$, then $y \in \I$.
\end{definition}

Note that, when $\P$ is finite, there is a one-to-one correspondence between
antichains $\A$ of $\P$ and order ideals $\I$. Namely, $\A$ is the set of maximal
elements of $\I$, while
\begin{equation}
\label{eq:poset:a}
\I = \left\{{x \in \P \st x \le z ~\text{for some}~ z \in \A}\right\}.
\end{equation}

The set of all order ideals of $\P$, ordered by inclusion, forms a poset denoted
$J(\P)$. If $\I$ and $\A$ are related as in \ref{eq:poset:a}, then we say that $\A$
generates $\I$. If $\A = \enum{z_1 , \ldots , z_k}$, then we
write $\I = \langle z_1 , \ldots , z_k \rangle$ for the order ideal generated by
$\A$. The order ideal $\langle z \rangle$ is the principal order ideal
generated by $z$, denoted $\Lambda_z$. Similarly $V_z$ denotes the principal
dual order ideal generated by $z$, that is, $V_z = \enum{x \in \P \st x \ge
z}$.

